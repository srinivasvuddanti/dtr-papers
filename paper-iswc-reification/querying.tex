\section{Querying}
\todo{introduction for this section}

\subsection{Syntax of {\SPARQLplus}}

	Let $\symAllTPsInfinite := (\symAllVariables \cup \symAllTPsInfinite \cup \symAllURIs) \times (\symAllVariables \cup \symAllURIs) \times (\symAllVariables \cup \symAllTPsInfinite \cup \symAllURIs \cup \symAllLiterals)$ where $\symAllVariables$ is a set of query variables that is disjoint from $\symAllURIs$, $\symAllBNodes$, and $\symAllLiterals$, respectively.
	%We assume a set $\symAllVariables$ of query variables that is disjoint from $\symAllURIs$, $\symAllBNodes$, and $\symAllLiterals$, respectively. Let $\symAllTPsInfinite := (\symAllVariables \cup \symAllTPsInfinite \cup \symAllURIs) \times (\symAllVariables \cup \symAllURIs) \times (\symAllVariables \cup \symAllTPsInfinite \cup \symAllURIs \cup \symAllLiterals)$.
A \emph{\TPplus} is a tuple $tp \in \symAllTPsInfinite$ for which exists a natural number $k$ such that $tp$ is $k$-nested\footnote{$k$-nestedness for $\symAllTPsInfinite$ is defined analog to $k$-nestedness for $\symAllTriplesInfinite$.}; to denote the set of all {\TPplus}s we write $\symAllTPs$ (i.e.~$\symAllTPs \subset \symAllTPsInfinite$).
	%To denote the set of all variables in a {\TPplus} $tp=(s,p,o)$ we write $\fctVars{tp}$; hence, $\fctVars{tp} := (\lbrace s,p,o \rbrace \cap \symAllVariables) \cup \big\lbrace ?v \in \fctVars{x} \,|\, x \in \lbrace s,o \rbrace \cap \symAllTPs \big\rbrace$.
	For any {\TPplus} $tp \!=\! (s,p,o)$ we define $\fctVars{tp} \!:=\! \bigl( \lbrace s,p,o \rbrace \cap \symAllVariables \bigr) \cup \big\lbrace ?v \in \fctVars{x} \,\big|\, x \in \lbrace s,o \rbrace \cap \symAllTPs \big\rbrace$.

	We now define \emph{{\SPARQLplus} expression}s recursively \removable{as follows}:
	%\emph{{\SPARQLplus} expression}s are defined recursively \removable{as follows}:
\begin{enumerate}
	\item A \emph{\TPplus} is a {\SPARQLplus} expression.
	\item Let $tp \in \symAllTPs$ and $?v \in \symAllVariables$, then $(tp \OpAS ?v)$ is a {\SPARQLplus} expression.
	\item Let $P_1$ and $P_2$ be {\SPARQLplus} expressions and $R$ a filter condition%
		\footnote{For a formal definition of
			\removable{the syntax and semantics of}
		filter conditions we refer to~\cite{Perez09:SemanticsAndComplexityOfSPARQL}.}%
	, then $(P_1 \OpAND P_2)$, $(P_1 \OpUNION P_2)$, $(P_1 \OpOPT P_2)$, and $(P_1 \OpFILTER R)$ are {\SPARQLplus} expressions.
	\item Let $P$ be a {\SPARQLplus} expression and $S \subset \symAllVariables$ be a finite set of variables, then $\OpSELECT_S(P)$ is a {\SPARQLplus} expression.
\end{enumerate}


\subsection{{\RDFplus} Semantics of {\SPARQLplus}}
	To define the semantics of {\SPARQLplus} we introduce
	%We define the semantics of {\SPARQLplus} based on
\emph{valuation}s, that are, partial mappings $\eta : \symAllVariables \rightarrow \symAllTriples \cup \symAllURIs \cup \symAllBNodes \cup \symAllLiterals$.
Two valuations $\eta$ and $\eta'$ are \emph{compatible}, denoted by $\eta \compatible \eta'$, if for all $?v \in \fctDom{\eta} \cap \fctDom{\eta'}$ holds $\eta(?v) = \eta'(?v)$; note,
	%$\fctDom{\eta}$ denotes
	$\fctDom{\eta}$~denotes %% Layout Adjustment
the domain of a valuation $\eta$.

Let $\Omega_1$, $\Omega_2$ and $\Omega$ be sets of valuations; let $R$ be a filter condition; and let $S \subset \symAllVariables$ be a finite set of variables. The algebra operators \emph{join}, \emph{union}, \emph{difference}, \emph{left outer-join}, \emph{selection}, and \emph{projection} are defined as follows:
\begin{align*}
	\Omega_1 \Join \Omega_2 & := \lbrace \eta_1 \cup \eta_1 \,|\, \eta_1 \in \Omega_1 \text{ and } \eta_2 \in \Omega_2 \text{ and } \eta_1 \sim \eta_2 \rbrace \\
	\Omega_2 \cup \Omega_2 & := \lbrace \eta \,|\, \eta \in \Omega_1 \text{ or } \eta \in \Omega_2 \rbrace \\
	\Omega_1 \setminus \Omega_2 & := \lbrace \eta_1 \in \Omega_1 \,|\, \forall \eta_2 \in \Omega_2 : \eta_1 \not\sim \eta_2 \rbrace \\
	\Omega_1 \LJoin \Omega_2 &:= \left( \Omega_1 \Join \Omega_2 \right) \cup \left( \Omega_1 \setminus \Omega_2 \right) \\
	\sigma_R ( \Omega ) &:= \lbrace \eta \in \Omega \,|\, \eta \text{ satisfies } R \rbrace \\
	\pi_S ( \Omega ) &:= \lbrace \eta \,|\, \exists \, \eta' : \eta \cup \eta' \in \Omega \text{ and } \fctDom{\eta} \subseteq S \text{ and } \fctDom{\eta'} \cap S = \emptyset \rbrace
\end{align*}

%For a valuation $\eta$ and a {\TPplus} $tp$, we write $\eta[tp]$ to denote the {\TPplus} that we obtain by replacing the variables in $tp$ (including potentially embedded {\TPplus}s) according to $\eta$. Note, if $\fctVars{tp} \subseteq \fctDom{\eta}$, then $\eta[tp]$ is an {\RDFplusTriple}.

	%Let $P$ be a {\SPARQLplus} expression and let $G$ be an {\RDFplusGraph}. The \emph{evaluation of $P$ over $G$},
	The \emph{evaluation} of {\SPARQLplus} expression $P$ over {\RDFplusGraph} $G$,
denoted by $\fctEvalPlus{P}{G}$, is defined recursively as follows:
\begin{enumerate}
	\item If $P$ is a {\TPplus} $tp$, then
		\begin{equation*}
			\fctEvalPlus{P}{G} := \big\lbrace \eta \,\big|\,
				\eta \text{ is a valuation with } \fctDom{\eta} = \fctVars{tp}
				\text{ and } \eta[tp] \in G \cup \fctTRefs{G}
			\big\rbrace
		\end{equation*}
		where $\eta[tp]$ denotes the {\TPplus} that we obtain by replacing the variables in $tp$ (including potentially embedded {\TPplus}s) according to $\eta$ (note, if $\fctVars{tp} \subseteq \fctDom{\eta}$, then $\eta[tp]$ is an {\RDFplusTriple}).
		\vspace{1ex} %% Layout Adjustment
	\item If $P$ is $(tp \OpAS ?v)$, then
		\begin{align*}
			\fctEvalPlus{P}{G} := \big\lbrace \eta \,\big|\,
				& \eta' \in \fctEvalPlus{tp}{G} 
				\text{ and } %\eta \text{ is a valuation with }
					\fctDom{\eta} = \fctDom{\eta'}\cup \lbrace ?v \rbrace
				\text{ and } \eta \compatible \eta' \text{ and }\\
				& \eta(?v) = \eta[tp] \,
			\big\rbrace
		\end{align*}
	\item If $P$ is $(P_1 \OpAND P_2)$, then $\fctEvalPlus{P}{G} := \fctEvalPlus{P_1}{G} \Join \fctEvalPlus{P_2}{G}$.
	\item If $P$ is $(P_1 \OpUNION P_2)$, then $\fctEvalPlus{P}{G} := \fctEvalPlus{P_1}{G} \cup \fctEvalPlus{P_2}{G}$.
	\item If $P$ is $(P_1 \OpOPT P_2)$, then $\fctEvalPlus{P}{G} := \fctEvalPlus{P_1}{G} \LJoin \fctEvalPlus{P_2}{G}$.
	\item If $P$ is $(P' \OpFILTER R)$, then $\fctEvalPlus{P}{G} := \sigma_R\bigl( \fctEvalPlus{P'}{G} \bigr)$.
	\item If $P$ is $\OpSELECT_S(P')$, then $\fctEvalPlus{P}{G} := \pi_S\bigl( \fctEvalPlus{P'}{G} \bigr)$.
\end{enumerate}
Each valuation $\eta \in \fctEvalPlus{P}{G}$ is called a \emph{solution} for $P$ in $G$.

\begin{proposition}
	For any {\RDFplusGraph} $G$ and any {\SPARQLplus} expression $P$ it holds $\fctEvalPlus{P}{G} = \fctEvalPlus{P}{G^+}$ where $G^+ = G \cup \fctTRefs{G}$.
\end{proposition}

\todo{There is a minimal $G_\mathsf{min} \subseteq G$ with $G_\mathsf{min} = G^+$ ...}


\subsection{RDF Semantics of {\SPARQLplus}}
	%For RDF storage systems that do not support our {\RDFplus} data model (but the ordinary RDF data model), we define an alternative semantics for {\SPARQLplus}.
	In addition to the {\SPARQLplus} semantics defined in the previous section, we now define an alternative semantics for {\SPARQLplus}. This alternative semantics enables RDF systems that support the ordinary RDF data model (but not our {\RDFplus} data model), to answer {\SPARQLplus} queries.

\todo{...}

GOAL: for all $P$ that do not contain a subexpression of the form $(tp \OpAS ?v)$ for which $?v \in pVars(\fctEvalPlus{P}{G})$ it shall hold: $\fctEvalPlus{P}{G} = \fctEvalPlus{\fctDescent{P}}{\fctDescent{G}} = \fctEval{\fctDescent{P}}{\fctDescent{G}}$. Note, the set of possible variable $pVars\bigl(\fctEvalPlus{P}{G}\bigr)$ of a {\SPARQLplus} algebra expression $\fctEvalPlus{P}{G}$ is defined in the same way Schmidt et al.~define $pVars$ for SPARQL set algebra expressions~\cite{Schmidt10:FoundationsOfSPARQLOptimization}, we only extend the definition for our $\OpAS$ expression as follows: $pVars\bigl(\fctEvalPlus{(tp \OpAS ?v)}{G}\bigr) := \lbrace ?v \rbrace \cup \fctVars{tp}$.

Let $P$ be a {\SPARQLplus} expression and let $V_\mathsf{new} \subseteq \symAllVariables \cap \fctVars{P}$.
\todo{we need a mapping from all triple reference patterns in $P$ to $V_\mathsf{new}$ so that we can use the same var all the time}
The \emph{RDF compatible expression} of {\SPARQLplus} expression $P$,
denoted by $\fctDescent{P}^V$, is defined recursively as follows:
\begin{enumerate}
	\item If $P$ is a {\TPplus} $tp$, then $\fctDescent{P}^V$ is \todo{...}
	\item If $P$ is $(tp \OpAS ?v)$, then $\fctDescent{P}^V$ is \todo{...}
	\item If $P$ is $(P_1 \OpAND P_2)$, then $\fctDescent{P}^V$ is $(P_1' \OpAND P_2')$ where $P_1'$ is $\fctDescent{P_1}^V$ and $P_2'$ is $\fctDescent{P_2}^V$.
	\item If $P$ is $(P_1 \OpUNION P_2)$, then $\fctDescent{P}^V$ is $(P_1' \OpUNION P_2')$ where $P_1'$ is $\fctDescent{P_1}^V$ and $P_2'$ is $\fctDescent{P_2}^V$.
	\item If $P$ is $(P_1 \OpOPT P_2)$, then $\fctDescent{P}^V$ is $(P_1' \OpOPT P_2')$ where $P_1'$ is $\fctDescent{P_1}^V$ and $P_2'$ is $\fctDescent{P_2}^V$.
	\item If $P$ is $(P_1 \OpFILTER R)$, then $\fctDescent{P}^V$ is $(P_1' \OpFILTER R)$ where $P_1'$ is $\fctDescent{P_1}^V$.
	\item If $P$ is $\OpSELECT_S(P_1)$, then $\fctDescent{P}^V$ is $\OpSELECT_S(P_1')$ where $P_1'$ is $\fctDescent{P_1}^V$.
\end{enumerate}