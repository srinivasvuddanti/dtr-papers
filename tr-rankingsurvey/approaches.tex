\section{Ranking Approaches}

\subsection{Historical Development} 


\subsection{Generic Architecture} 


\subsection{Main Dimensions} 
Features: content, structure, context, two meanings of context: context of user; and context of the data such as trust values of source, truth value / confidence degree of individual records; make clear this survey does not discuss user-context based methods in details\\
Results: entity, relationships, subgraphs, graphs (entire dataset)\\
Techniques:\\
- may vary in terms of methods: used \emph{NLP} for understanding keywords, used \emph{ML} for classifying keywords (I know some ML-based query classification approaches for document retrieval, are there also examples for structured data ranking?) and for learning to rank, (other) statistics-based methods: Vector Space Model (VSM), Language Modeling (LM), Information Theory\\
- may vary in terms of heuristics: two main ones are query-relevance and popularity; other heuristics are proximity, informativeness (based on Information Theory, e.g. entropy) and context-based: trust, truth value, locality etc. 


\subsection{Foundation}
Here we discuss all basics needed to understand the approaches presented in the following sections.\\
Vector-Space Model: also discuss pivoted normalization and point out problems of short document length etc. in the context of structured data\\
Language Modeling: also discuss smoothing strategies\\
Link Analysis\\
Learning to Rank\\
... 

\subsection{Taxonomy of Ranking Approaches}
Classify approaches mainly based on the type of \emph{heuristics} they used. Those that use same heuristics are distinguished in terms of \emph{methods}. E.g. Query-relevance based solutions can be further distinguished in terms of VSM-based and LM-based approaches.\\

 
