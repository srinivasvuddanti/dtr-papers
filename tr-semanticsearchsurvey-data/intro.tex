\section{Introduction}
% unsolved information needs%
Advancements in search technologies enable users to deal with and to exploit the ever increasing amount of information that we can find in different settings -- from the personal desktop to enterprises' databases up to the large-scale Web. As opposed to the structured querying capabilities commonly provided by commercial databases that are only accessible by experts, \emph{search} is rather understood as an end-user oriented paradigm that is based on intuitive interfaces and access mechanisms. Solutions that are widely used especially on the Web are based on keyword search, and in smaller-scale scenarios, also support natural language (NL) inputs. While they are easy to use, the inputs provided by the users through these interfaces are \emph{ambiguous} such that the underlying information need (query intent) is not directly accessible to the system. Further, there are search scenarios where not only the query, but also the data representation does not precisely capture the semantics and structure of the underlying information. Accordingly, the core problems to be addressed in search are (1) to \emph{understand the query intent and the data}, (2) to \emph{match query against data}, and (3) to \emph{rank results}. 

Modern search engines are efficient in retrieving documents (web pages in the case of web search) from large collections based on matching the text of the user's query with the textual content of the document or words in linking to the current document (\emph{anchortext}). These textual features are often combined with additional features that measure the popularity of results based on usage data (\emph{click-behaviour} or using metrics computed on the web graph such as \emph{PageRank} and its derivates. In combination, these relatively simple, but scalable matching techniques are are able to address a large fraction of common information needs. They work particularly well when the user is able to explicitly name the target of the query, this reference is unambiguous and refers to a popular item widely referred to with that name. An example is a product search, where the user enters the name of the product and that product is widely known by that name, e.g \emph{canon ixus 115}, a digital camera.


\subsection{Many Unsolved Long-tail Queries}

While search engines have been effective in addressing common needs, most queries are still in the ``long tail'' of Web query logs. Baeza-Yates et al. report that even in a full year of query logs 88\% of the unique queries are singleton queries, i.e. appeared only once in the year. A lot of this variety comes from syntactic variations of the same popular need and can be addressed with spell-corrections and query suggestion techniques. However, the remaining queries tend to be less common because they refer to less popular needs that need disambiguation or require more complex explanation.

%\begin{itemize}
%	\item 
	\textbf{Ambiguous Queries:} 
	Ambiguous queries arise naturally when the user may not be able to name precisely the item that is sought. Coming back to the example of product search, a user who is in the early stages of selection may not know exactly which product to look for. In this case, the user may name one or two characteristics of the product, such as \emph{cheap digital camera}. The concept of digital camera covers a broad range of actual entities, and there might be some discussion as to which of these can be considered cheap.There are also systemic factors that lead to an increase in ambiguity. Despite the expectation that longer queries should leave to better results, current search engines perform poorer on longer specifications of the same information need, largely because search engines employ an AND semantics between words, and the space of documents that contain all the words becomes smaller with every word added and determining the important words becomes more difficult. Thus users have learned to avoid long queries, leading to shorter but more ambiguous queries. Lastly, users will try to find what they are looking for with the least possible words, adding additional terms after realizing that their original query was ambiguous. For example, when searching for people the user may type in a name, only to realize that there are multiple persons with the same name. Often, this is aggravated by the bias toward a single popular answer. For example, when searching for 'George Bush', the top results are crowded by references to the same person, even though this is a common name. \footnote{For other notable persons named George Bush, see \url{http://en.wikipedia.org/wiki/George_Bush_\%28disambiguation\%29}}.
			
%	\item 
	\textbf{Complex Queries:} We include in this class all queries that go beyond a reference to a single named entity, and involve several entities and relationships between them. These queries arise again when the user does not know the answer to a question and therefore has to describe the item sought and it's relationships to other items that he or she can name. For example, searching for this paper by the name of the second author is relatively straightforward. However, assuming the user does not know (or do not remember) the name, he or she might look for a "semantic search survey paper written by a researcher at yahoo". Complex queries are particularly hard to address in the traditional information retrieval paradigm when they require information extraction and aggregation, i.e. when not all the entities and relationships are mentioned in a single document. Queries for events naturally involve multiple entities, typically the participant in the event and a reference to another item, e.g. \emph{barrack obama birth certificate}. 
%\end{itemize}

Clearly, these two categories represent orthogonal aspects, namely the ambiguity of query terms and the complexity of the intended information need. Many queries in the long tail are both ambiguous and complex. They require \emph{deep understanding} of the need and the information behind the query and data, respectively. 	

\subsection{Large Amount of Semantic Data}
Targeting these problematic cases, there is a category of search solutions -- often referred to as semantic search solutions -- that aim at exploiting the large and increasing amount of semantic resources that have been made available in different settings. Prominently, Google refers to this as the next generation of search in a recent issue of the Wall Street Journal.

Semantic resources are used by these solutions for the interpretation of queries and data. They include taxonomies, thesauri and formal ontologies. 
More generally, also structured data exposed as graph-structured RDF\footnote{RDF stands for Resource Description Framework, a W3C Semantic Web standard for data representation and exchange on the Web: \url{http://www.w3.org/RDF/}} data are regarded as semantic resources (often called \emph{semantic data}). They come as standalone RDF datasets or are embedded in Web pages in the form of RDFa (a W3C standard\footnote{http://www.w3.org/TR/xhtml-rdfa-primer/} for embedding RDF data in Web pages). These data have been increasing rapidly. As a result of community projects such as Linking Open Data\footnote{\url{http://www.w3.org/wiki/SweoIG/TaskForces/CommunityProjects/LinkingOpenData}}, a large amount of structured datasets formerly kept protected in databases are now exposed as publicly Web-accessible RDF data (e.g. Linked Data\footnote{\url{http://linkeddata.org/}}). Besides large companies such as BestBuy and Facebook, also national governments such as the US and UK have followed this direction of publishing and linking semantic data on the Web. As a result of active adoption and support from Web search engines providers such as Google and Yahoo!, 10 percent of all Web pages are estimated to contain some form of semantic data markups such as RDFa~\cite{}. 

The main idea behind semantic search solutions is to use these semantic resources to improve the search performance. Taxonomies, thesauri, and ontologies capture semantics that can be used to understand the query and data and to \emph{resolve ambiguities}. Instead of returning textual data (Web pages, documents), precise facts capturing entity related information and their relationships can be directly retrieved from semantic data to provide \emph{direct answers to complex queries}. 

\subsection{Plethora of Semantic Search Solutions}
Capitalizing on the opportunities that arise from this large and increasing amount of semantic resources on the Web, commercial search engines have made different kinds of semantic search features available to end users. The first well-known example towards this direction is Powerset\footnote{\url{http://en.wikipedia.org/wiki/Powerset_(company)}}, which makes use of semantic data extracted from Wikipedia to answer complex questions. Semantic data embedded in Web pages are now actively used by Google to provide rich result snippets\footnote{\url{http://www.google.com/webmasters/tools/richsnippets}}. 

Likewise, research interest in semantic search is stirring up. A wide range of semantic search approaches have been proposed by researchers from different communities, including database, Information Retrieval (IR) and Semantic Web. Correspondingly, semantic search has been viewed from different angles, resulting in a plethora of solutions. 

Underlying all these efforts lie one central idea, namely to use semantic resources for more effective search. However, existing approaches greatly vary in the \emph{data}, (2) the \emph{semantic resources}, the \emph{information needs} and (4) the \emph{querying paradigms} that are supported. Most notably, there are principle differences between semantic search approaches, which use semantic resources to improve document retrieval, and the ones that compute direct results, i.e. semantic data retrieval. That is, whereas the one category of approaches focuses on textual data, the other category targets the case of structured and semantic data. The specific types of semantic resources used by these approaches vary and may include thesauri, full-fledge ontologies or large amounts of semantic data. With respect to information needs, solutions focusing on the retrieval of single entities (and documents) can be distinguished from relational ones, where relationships between entities have to be considered. Substantial differences also exist between systems, which either use NL, keywords or facet-based operations as the primary means for querying. Further, research work found in literatures often studies one of the specific (5) \emph{subproblems in search}: Namely, different concepts and techniques have been proposed for (5a) the interpretation of query inputs and data, (5b) matching the query intent against data and (5c) ranking results.  

\subsection{Contributions}
Different solutions for semantic search have been around for a long time. In this work, we present a systematic survey to provide a taxonomy of various semantic search approaches, to reflect on the achievements that have been made over these years, and finally, to discuss the challenges ahead and directions for future research. 

% and their mainstream adoption into commercial solutions is increasingly visible. 
There exist a few usability studies, which compare several semantic search solutions~\cite{DBLP:conf/semweb/KaufmannB07,DBLP:conf/esws/TranMH10}. However, this line of work focuses on the differences in querying paradigms. Towards a classification of semantic search approaches, Mangold discusses existing solutions along the dimensions of architecture, coupling, transparency, user context, ontology structure and ontology technology~\cite{DBLP:journals/ijmso/Mangold07}. The last two criteria basically capture the differences in semantic resources among existing systems, which are also discussed in our study. In addition to query paradigms and semantic resources, our study includes the aspects of data, information needs, and provides a fine-grained taxonomy that distinguishes approaches by the subproblems in search they addressing. Further, whereas the previous survey~\cite{DBLP:journals/ijmso/Mangold07} focuses on document retrieval, our study aims at a unified view of document and data retrieval. In particular, we cover a large body of recent work that is not part of that previous survey.  


%	\item Besides this general overview, we discuss the \emph{state-of-the-art in semantic data retrieval}, a direction of semantic search that aims at direct results as answers. For this, different data management and processing tasks are needed, from the crawling and integration of data up to the computation and presentation of results. 
%	\item We present specific solutions that have been proposed for these individual tasks. Particular emphasis is put on the most commonly used interface, namely \emph{keyword search}, and the concepts and techniques for dealing with the core search tasks of \emph{matching and ranking}. 
 


\subsection{Outline}
The paper is organized as follows: An overview of semantic search and the concepts it entails is presented in Section~\ref{sec:ss}. Then, the paper is organized mainly along the aspects based on which we distinguish different semantic search approaches. Different types of information needs, querying paradigms, data and semantic resources are presented in Section~\ref{sec:needs},~\ref{sec:querying},~\ref{sec:data} and ~\ref{sec:semantics}. Then approaches are discussed along the subproblems of data interpretation, query interpretation, matching and ranking in Section~\ref{sec:content},~\ref{sec:query},~\ref{sec:matching} and~\ref{sec:ranking}. After that, we present the main types of semantic search approaches and discuss their performances in Section~\ref{sec:approaches}. Main challenges derived from that and implications for future work are presented in Section~\ref{sec:challenges}, before we finally conclude in Section~\ref{sec:conclusion}.