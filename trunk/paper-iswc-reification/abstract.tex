Meta-level information is commonly associated with RDF statements to capture provenance as well as quality related aspects. RDF reification is the standard mechanism to express this kind of information at the statement-level. In this work, we revisit the drawbacks of this mechanism. Specifically in the RDF data management and query processing context, we show that not only it results in a large number of triples but also, requires a large number of joins in queries that ask for meta-level information. To address these problems we propose a solution for reification that extends the current RDF and SPARQL standards. Based on two commercial RDF stores' adoption of this solution, we show in experiments that our proposal significantly improves the efficiency of RDF data management and SPARQL query processing, compared to the original RDF reification proposal.