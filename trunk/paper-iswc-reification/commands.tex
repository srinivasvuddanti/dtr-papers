%%% review commands:
%%%%%%%%%%%%%%%%%%%%%%%%%%%%

\newcounter{todo}

\newcommand{\todo}[1]
	{\stepcounter{todo}\noindent{\color{red} \textsf{Todo~{\arabic{todo}}: #1}}}  % use this for drafts
	%{}                                                                           % use this for the final version

\newcommand{\Todo}[2]
	{{\color{red}\textsf{#1}\footnote{\todo{#2}}}}  % use this for drafts
	%{#1}                                           % use this for the final version

\newcommand{\removable}[1]%
	{{\color{gray}#1}}  % use this for drafts
	%{#1}               % use this for the final version



%%% key terminology:
%%%%%%%%%%%%%%%%%%%%%%%%%%%%

\newcommand{\xxxPLUS}% % don't use that; it is required to define \RDFplus and \SPARQLplus
	%{$^{\mathsf{\text{\tiny +\!meta}}}$}
	{$^\star$}

\newcommand{\RDFplus}{RDF\xxxPLUS}
\newcommand{\SPARQLplus}{SPARQL\xxxPLUS}

\newcommand{\RDFplusTriple}{{\RDFplus}\! triple}
\newcommand{\RDFplusGraph}{{\RDFplus}\! graph}
\newcommand{\TPplus}{triple\xxxPLUS pattern}

\newcommand{\TRef}%
	%{triple reference}
	{tref}
\newcommand{\TRefLongVersion}{triple reference}

\newcommand{\TRefPtt}{{\TRef} pattern}
\newcommand{\TRefPttLongVersion}{{\TRefLongVersion} pattern}



%%% relevant math symbols
%%%%%%%%%%%%%%%%%%%%%%%%%%%%

\renewcommand{\emptyset}{\varnothing}

\newcommand{\symDomFct}{\mathrm{dom}} % the name used for \fctDom
\newcommand{\fctDom}[1]{\symDomFct(#1)} % the domain of function/mapping #1

\newcommand{\symAllURIs}{\mathcal{U}} % the symbol for the set of all URIs
\newcommand{\symAllLiterals}{\mathcal{L}} % the symbol for the set of all literals
\newcommand{\symAllBNodes}{\mathcal{B}} % the symbol for the set of all blank nodes
\newcommand{\symAllVariables}{\mathcal{V}} % the symbol for the set of all query variables

	\newcommand{\symAllTriplesInfinite}{\mathcal{T}} % don't use that; it is required to define \symAllTriples
\newcommand{\symAllTriples}{\overline{\symAllTriplesInfinite}} % the symbol for the set of all triple references
	\newcommand{\symAllTPsInfinite}{\mathcal{TP}} % don't use that; it is required to define \symAllTPs
\newcommand{\symAllTPs}{\overline{\symAllTPsInfinite}} % the symbol for the set of all triple references

	\newcommand{\symAllTRefsInfinite}{\mathcal{R}} % don't use that; it is required to define \symAllTRefs
\newcommand{\symAllTRefs}{\overline{\symAllTRefsInfinite}} % the symbol for the set of all triple references
	\newcommand{\symAllTRefPttsInfinite}{\mathcal{RP}} % don't use that; it is required to define \symAllTRefPtts
\newcommand{\symAllTRefPtts}{\overline{\symAllTRefPttsInfinite}} % the symbol for the set of all triple reference patterns

\newcommand{\symTermsFct}{\mathrm{terms}} % the name used for \fctTerms
\newcommand{\fctTerms}[1]{\symTermsFct(#1)} % the set of all terms in triple #1

\newcommand{\symTermsPlusFct}{\symTermsFct^+} % the name used for \fctTermsPlus
\newcommand{\fctTermsPlus}[1]{\symTermsPlusFct\!(#1)} % the set of all terms in triple #1 and its embedded triples

\newcommand{\symTRefsFct}{\mathrm{trefs}} % the name used for \fctTrefs
\newcommand{\fctTRefs}[1]{\symTRefsFct(#1)} % the set of all triple references in triple/graph #1

% \newcommand{\symDerefFct}{\mathrm{deref}} % the name used for \fctDeref
% \newcommand{\fctDeref}[1]{\symDerefFct(#1)} % the triple denoted by triple reference #1

% \newcommand{\symCloseFct}{\mathrm{cl}} % the name used for \fctClose
% \newcommand{\fctClose}[1]{\symCloseFct(#1)} % the tref-closure of #1

\newcommand{\symVarsFct}{\mathrm{vars}} % the name used for \fctVars
\newcommand{\fctVars}[1]{\symVarsFct(#1)} % the set of all variables in pattern #1

\newcommand{\symDescentFct}{\mathrm{rdf}} % the name used for \fctDescent
\newcommand{\fctDescent}[1]{\symDescentFct(#1)} % set of RDF triples for an RDF+ triple/graph

\newcommand{\mathTinyRDF}[1]{\texttt{\footnotesize #1}}

\newcommand{\OpAND}{\text{ \normalfont\scriptsize\textsf{AND} }}
\newcommand{\OpUNION}{\text{ \normalfont\scriptsize\textsf{UNION} }}
\newcommand{\OpOPT}{\text{ \normalfont\scriptsize\textsf{OPT} }}
\newcommand{\OpFILTER}{\text{ \normalfont\scriptsize\textsf{FILTER} }}
\newcommand{\OpAS}{\text{ \normalfont\scriptsize\textsf{AS} }}
\newcommand{\compatible}{\sim} % the symbol for compatibility between valuations
\newcommand{\fctEval}[2]{[\![#1]\!]_{#2}} % the symbol for the evaluation of SPARQL pattern #1 over an RDF graph #2

%% The following definitions of a symbol for left outer join work best in Springer format:
\def\ojoin{\setbox0=\hbox{$\Join$}\rule[.01ex]{.25em}{.45pt}\llap{\rule[1.05ex]{.25em}{.4pt}}}
\def\LJoin{\mathbin{\ojoin\mkern-7.5mu\Join}}