\section{Experiments}
We now study the advantages of our approach as discussed previously through an experimental analysis, in which we compare our work with an implementation of the standard proposal for RDF reification. The results suggest... 

\subsection{Data}
-	dataset description: which dataset, which version, how did we generate meta-level statements, what kinds\\
- dataset statistics: how many triples, how many meta-level triples\\

\subsection{Queries}
-	query set description: which queries, how did we generate meta-level queries, what kinds\\
- query statistics: average, min, max number of triple patterns / meta-level triple patterns per query\\ 
- group queries according to complexity classes: (1) according to overall number of triple patterns, e.g. 1,2,3 and so on (2) according to number of meta-level triple patterns, (3) according to data size / selectivity (measured in terms of the total number of triples that have to be loaded), e.g. 1-1000 triples, 1000-1000, and so on\\

\subsection{Systems}
- Description of Bigdata: what is that, how does it implement our proposal: h

Virtuoso

\subsection{Data Management Performance}


\subsection{Query Processing Performance}

Show the benefits in experiments
Systems

Data
-	Size of the data
-	Size of the indexes 
Loading Time Performance
Querying Time Performance

