\documentclass{llncs}

\usepackage{amsmath,amssymb}
% \usepackage[ruled,vlined]{algorithm2e}
\usepackage{color}  \definecolor{gray}{rgb}{.5,.5,.5}
\usepackage{times}
% \usepackage{graphicx}
% \usepackage{subfig}
% \usepackage{epsfig}
% \usepackage{url}
% \usepackage[utf8]{inputenc}
% \usepackage{paralist}
%\usepackage{subfig}
%\usepackage{algorithm}
%\usepackage{algorithmic}
% \usepackage{makeidx} % allows for indexgeneration
%\usepackage{natbib}
%\usepackage[top=3.95cm, bottom=4cm, left=4.6cm, right=4.6cm]{geometry}


% \usepackage{latexsym}


%%% review commands:
%%%%%%%%%%%%%%%%%%%%%%%%%%%%

\newcounter{todo}

\newcommand{\todo}[1]
	{\stepcounter{todo}\noindent{\color{red} \textsf{Todo~{\arabic{todo}}: #1}}}  % use this for drafts
	%{}                                                                           % use this for the final version

\newcommand{\Todo}[2]
	{{\color{red}\textsf{#1}\footnote{\todo{#2}}}}  % use this for drafts
	%{#1}                                           % use this for the final version

\newcommand{\removable}[1]%
	{{\color{gray}#1}}  % use this for drafts
	%{#1}               % use this for the final version



%%% key terminology:
%%%%%%%%%%%%%%%%%%%%%%%%%%%%

\newcommand{\RDFplus}%
	%{RDF$^{\mathsf{\text{\tiny +\!meta}}}$}
	{RDF$^\star$}

\newcommand{\RDFplusTriple}{{\RDFplus}\! triple}
\newcommand{\RDFplusGraph}{{\RDFplus}\! graph}

\newcommand{\TRef}%
	%{triple reference}
	{tref}
\newcommand{\TRefLongVersion}{triple reference}


%%% relevant math symbols
%%%%%%%%%%%%%%%%%%%%%%%%%%%%

\renewcommand{\emptyset}{\varnothing}

\newcommand{\symAllURIs}{\mathcal{U}} % the symbol for the set of all URIs
\newcommand{\symAllLiterals}{\mathcal{L}} % the symbol for the set of all literals
\newcommand{\symAllBNodes}{\mathcal{B}} % the symbol for the set of all blank nodes
\newcommand{\symAllVariables}{\mathcal{V}} % the symbol for the set of all query variables
	\newcommand{\symAllTRefsInfinite}{\mathcal{R}} % don't use that; it was required to define \symAllTRefs
\newcommand{\symAllTRefs}{\overline{\symAllTRefsInfinite}} % the symbol for the set of all triple references

\newcommand{\symTermsFct}{\mathrm{terms}} % the name used for \fctTerms
\newcommand{\fctTerms}[1]{\symTermsFct(#1)} % the set of all terms in triple/graph #1

\newcommand{\symTRefFct}{\mathrm{trefs}} % the name used for \fctTref
\newcommand{\fctTRef}[1]{\symTRefFct(#1)} % the set of all triple references in triple/graph #1

\newcommand{\symDerefFct}{\mathrm{deref}} % the name used for \fctDeref
\newcommand{\fctDeref}[1]{\symDerefFct(#1)} % the triple denoted by triple reference #1

\newcommand{\symDescentFct}{\mathrm{rdf}} % the name used for \fctDescent
\newcommand{\fctDescent}[1]{\symDescentFct(#1)} % set of RDF triples for an RDF+ triple/graph

\newcommand{\mathTinyRDF}[1]{\texttt{\footnotesize #1}}

%\renewcommand{\baselinestretch}{0.98}
% \frontmatter          % for the preliminaries
% \pagestyle{headings}  % switches on printing of running heads
% \mainmatter           % start of the contributions

%\sloppy
\title{Efficient Management of Reified Statements in RDF}

%\thanks{This work has been supported by the European Commission
%under contract IST-2006-027595 NeOn, and by the Deutsche
%Forschungsgemeinschaft (DFG) under the ReaSem project.}}


%\titlerunning{}  % abbreviated title (for running head)
%                                     also used for the TOC unless
%                                     \toctitle is used
%
\author{Olaf Hartig\inst{1} \and Mike Personick\inst{2} \and Yrj{\"a}n{\"a} Rankka\inst{3} \and Bryan Thompson\inst{2} \and Thanh Tran\inst{4}}

\authorrunning{Hartig}   % abbreviated author list (for running head)

%%% modified list of authors for the TOC (add the affiliations)
\tocauthor{}

\institute{%
Humboldt-Universit{\"a}t zu Berlin, Berlin, Germany\\
\and
SYSTAP LLC, Washington, USA\\
\and
OpenLink Software, Burlington, USA\\
\and
Institute AIFB, Karlsruhe Institute of Technology, Germany
}

\makeindex

\begin{document}

\maketitle

\begin{abstract}
Metadata is commonly associated with RDF statements to capture provenance as well as quality related aspects. RDF reification is the standard mechanism to express this kind of information at the statement-level. In this work, we revisit the drawbacks of this mechanism. Specifically in the RDF data management and query processing context, we show that not only it results in a large number of triples but also, requires a large number of joins in queries asking for RDF statements' metadata. Addressing these problems, we propose a solution for reification that extends the current RDF and SPARQL standards. Based on two commercial RDF stores adoption of this solution, we show in experiments that compared to the original RDF reification proposal, it significantly improves the efficiency of RDF data management and SPARQL query processing. 
\end{abstract}

\section{Introduction}
% unsolved information needs%
Advancements in search technologies enable users to deal and to exploit the ever increasing amount of information that we can find in different scenarios -- from the personal desktop to enterprises' databases up to the large-scale Web. As opposed to the structured querying capabilities commonly provided by commercial databases that are only accessible by experts, \emph{search} is rather understood as a \emph{end-user oriented paradigm} that is based on \emph{intuitive interfaces} and access mechanisms. Solutions that are widely used especially on the Web are based on \emph{keyword search}, and in smaller-scale scenarios, also support natural language (NL) inputs. While they are easy to use, the inputs provided by the users through these interfaces are \emph{ambiguous} such that the underlying information need (query intent) is not directly accessible to the system. Further, there are search scenarios where not only the query, but also the data representation does not precisely capture the semantics and structure of the underlying information. Accordingly, the core problems to be addressed in search are to \emph{understand the query intent and the data}, to \emph{match query against data}, and to \emph{rank results}. 

\subsection{Many Unsolved Long-tail Queries}
Modern search engines can effectively interpret the topics and based on that, disambiguates the need intended in the provided query inputs. Combining with additional (query-independent) factors such as PageRank, they are able to address a large fraction fraction of common information needs. In particular, queries that request for pages about a particular topic (\emph{topical query}) or representing popular entry points to commonly used information that can be reached via further browsing activities (\emph{navigational queries}), can be successfully processed. However, dealing with the large number of distinct queries in the ``long tail'' of Web query logs \cite{} remains challenging. Two particular categories of problematic cases are ambiguous and complex queries:

\begin{itemize}
	\item \emph{Ambiguous queries}: these queries contain ambiguous terms, resulting in many possible interpretations (several query intents). For instance, the query ``young scholars from Germany'' is difficult to processed because without any additional context, it is not clear who is ``young'', and different types of people may be considered as ``scholars''. 

	\item \emph{Complex queries}: this class of queries capture complex information needs which may involve several entities and relationships between them. As an example, we have a query that mentions a person and a location, and asks for a relationship between that person and another location (relational queries):  ``birthplace of 32 year old computer scientist, who lives in Karlsruhe, a city in Germany''. 
\end{itemize}

Clearly, these two categories represent orthogonal aspects, namely the \emph{ambiguity of query terms} and the \emph{complexity of the intended information need}. Many queries in the long tail are both ambiguous and complex. They require deep understanding of the need and the information behind the query and data, respectively. 	

\subsection{Large Amount of Semantic Data}
Targeting these problematic cases, semantic search solutions aim at exploiting the large and increasing amount of semantic resources that have been made available in different settings. These resources include taxonomies, thesauri and formal ontologies that can be used for the interpretation of query terms and data representation. More generally, also structured data exposed as graph-structured RDF data belong to this type of semantic resources (henceforth called \emph{semantic data}) that are used for semantic search. They come as standalone RDF datasets or are embedded in Web pages in the form of RDFa. These data have been increasing rapidly. As a result of community projects such as Linking Open Data\footnote{}, a large amount of structured datasets formerly kept protected in databases are now exposed as publicly Web-accessible RDF data. Besides large companies such as BestBuy and Facebook, also national governments such as the US and UK have followed this direction of publishing and linking semantic data on the Web. As a result of active adoption and support from Web search engines providers such as Google and Yahoo!, 10 percent of all Web pages are estimated to contain some form of semantic data markups such as RDFa~\cite{}. 

The main idea behind semantic search solutions is to use these semantic resources to improve the search performance. Taxonomies, thesauri, and ontologies capture semantics that can be used to understand the query and data and to resolve ambiguities. Instead of returning textual data in documents, precise facts capturing entity related information and their relationships can be directly retrieved from semantic data to provide \emph{direct answers} to complex queries. 

\subsection{Plethora of Semantic Search Solutions}
Capitalizing on the opportunities that arise from this increasing wealth of semantic resources on the Web, commercial search engines have made different kinds of semantic search features available to end users. The first well-known example towards this direction is Powerset, which makes use of semantic data extracted from Wikipedia to answer complex questions. Semantic data embedded in Web pages are now actively used by Google to provide rich result snippets\footnote{}. 

Likewise, research interest in semantic search is stirring up. A large body of work has been proposed by researchers from different communities, including database, Information Retrieval (IR) and Semantic Web. Correspondingly, semantic search has been viewed from different angles, resulting in a plethora of solutions. While underlying all these efforts lie one central idea, namely to use semantic resource for more effective search, existing approaches greatly vary in the information needs, the query and the data that are supported. Hence, different concepts and techniques have been studied in these settings. Most notably, there are principle differences between semantic search approaches, which use semantic resource to improve document retrieval, and the ones that compute direct results, i.e. \emph{semantic data retrieval}. 

\subsection{Contributions}
Concepts for semantic search have been around for several several years and its mainstream adoption into commercial solutions is increasingly visible. 

\begin{itemize}
	\item In this work, we present a systematic survey to provide a \emph{taxonomy of various semantic search approaches}, to reflect on the achievements that have been made over these years, and finally, to discuss the remaining challenges and directions for future research. To the best of our knowledge, the only work... 
	\item Besides this general overview, we discuss the \emph{state-of-the-art in semantic data retrieval}, a direction of semantic search that aims at direct results as answers. For this, different data management and processing tasks are needed, from the crawling and integration of data up to the computation and presentation of results. 
	\item We present specific solutions that have been proposed for these individual tasks. Particular emphasis is put on the most commonly used interface, namely \emph{keyword search}, and the concepts and techniques for dealing with the core search tasks of \emph{matching and ranking}. 
 
\end{itemize}


\subsection{Outline}

\section{Related Work} \label{Section:RelWork}
Many proposals for extending Semantic Web standards to support different metadata scenarios exist. 

For data captured in OWL ontologies, a strict separation of domain data and metadata at the level of semantics has been proposed~\cite{DBLP:conf/aaai/TranHMGH08}. Interpreting these two types of data together can lead to undesirable effects on the semantics and in particular, introduce large overhead in reasoning and answering queries that involve both types of data. Based on the observation that these two types have distinct universes of discourses, it has been proposed to represent metadata as axiom annotations, ontology elements akin to comments that do not affect the semantics of domain data. Semantics is then given to metadata by translating it to a metaview, an ontology capturing the semantics of metadata that is interpreted independently from the ontology about domain data.  

Several extensions have also been proposed for RDF to support some specific types of metadata. Using Fuzzy RDF~\cite{DBLP:conf/rr/Straccia09} for instance, a degree of truth can be assigned to RDF triples such as \verb+(_:bob foaf:mbox <mailto:bob@home>):0.9+. Further works in this direction~\cite{swap2004,DBLP:conf/semweb/MazzieriD05} focus on the semantics of metadata, basically answering the question of what it means to have triples associated with some truth values and how to use them for reasoning by the deductive system. For dealing with temporal metadata, Temporal RDF as been proposed, which is equipped with semantics and a deductive system that can be reduced to the standard semantics of and inferences over RDF graphs~\cite{DBLP:journals/tkde/GutierrezHV07}. 

There are also several extensions to SPARQL for querying metadata. Hartig proposes tSPARQL for querying trust values of results that can be found for graph patterns specified in SPARQL queries~\cite{DBLP:conf/esws/Hartig09}. Besides extensions (such as tSPARQL) that introduce specific SPARQL constructors to access a particular type of metadata, there exists also a framework for querying metadata that generalizes over the many possible types of metadata~\cite{DBLP:conf/www/SchuelerSST08}. It defines a \verb+WITH META+ clause that can be used in SPARQL queries to specify the graphs for which metadata shall be retrieved. Since different types of metadata might be associated with such graphs, this framework enables querying over metadata in a generic fashion.  

Generalizing over RDF and SPARQL extensions that address a particular type of metadata, a general framework for representing, querying and reasoning with metadata has been proposed recently~\cite{DBLP:journals/ws/ZimmermannLPS12}. As in previous works such as Fuzzy RDF, it uses an abstract syntax for representing metadata: a triple annotated with metadata is represented as $\tau: \lambda$ where $\tau$ is a triple and $\lambda$ is an annotation value. The RDFS semantics is then extended to provide a deductive system for
this kind of annotated triples, called the Annotated RDFS semantics. In order to accommodate for the different possible semantics of annotation values, e.g. temporal, fuzzy or provenance annotation values, a general structure based on semiring is used as the interpretation domain. Finally, AnQL~\cite{DBLP:conf/semweb/LopesPSZ10} is introduced as an SPARQL extension for querying metadata. Similar to the abstract syntax for annotated triples, an AnQL query is constructed based on annotated triple patterns of the form  $\eta : \pi$ where $\eta$ is a triple pattern and $\pi$ is either an annotation value or an annotation variable. 

These previous works are focused on the semantics of metadata, providing corresponding extensions to the RDF(S) semantics to enable reasoning and querying over different types of metadata. However, they do not directly address the problem of representing metadata in RDF. In fact, RDF reification and named graphs are still the main mechanisms that underly these approaches. For instance, Fuzzy RDF statements represented in the abstract syntax, e.g. \verb+(_:bob foaf:mbox <mailto:bob@home>):0.9+, are translated into RDF triples by using RDF reification~\cite{swap2004}. Metadata used in the querying framework discussed before are assumed to be represented as named graphs~\cite{DBLP:conf/www/SchuelerSST08}. Orthogonal to existing works on the semantics of metadata, we provide an efficient RDF reification mechanism. The SPARQL extension that goes along with it is specifically needed to query metadata captured by this mechanism. It can be used as the basis for supporting the more specific query semantics as proposed for different types of metadata. 




\section{Discussion}
What are benefits compared to RDF reification and other related work:
Compact Representation 
Efficient Querying 
 # Who bought Sybase and where did we get that fact?

SELECT ?src ?who {
   <<?who :bought :sybase>> dc:source ?src
}

Answered efficiently
-	2-bound POS index scan (?who :bought sybase) => ?sid
-	Join with (?sid dc:source ?src)

# Explicit bind of statement on variable for multiple joins.

SELECT ?src ?who ?created {
   <<?who :bought :sybase>> as ?sid
   ?sid dc:source ?src
   OPTIONAL {?sid dc:created ?created}
}

Compatibility with Existing Standards
-	Harmonized with RDF Reification
-	Can be combined with named graphs


\section{Data Model} \label{Section:DataModel}

In this section we introduce
	%an extension of the RDF data model~\cite{Klyne04:RDFconcepts} that makes 
	our data model which we call {\RDFplus}. This model is an extension of the RDF data model~\cite{Klyne04:RDFconcepts}. {\RDFplus} extends RDF by making metadata statements a first class citizen in the data model.
% In the following we \todo{...}

% Main purposes of our data model: 1.)~it allows for a more compact representation; 2.)~it allows for efficient processing
% with {\RDFplus} our goal is to extend/expand the applicability/usefulness/usability of such triples for cases where ...


\subsection{Concepts}

We assume
	%\removable{three,}
pairwise disjoint sets $\symAllURIs$ (all URI references), $\symAllBNodes$ (blank nodes), and $\symAllLiterals$ (literals).
	As usual, an \emph{RDF triple} is a tuple $(s,p,o) \in (\symAllURIs \cup \symAllBNodes) \times \symAllURIs \times (\symAllURIs \cup \symAllBNodes \cup \symAllLiterals)$ and an \emph{RDF graph} is a set of RDF triples.
	%As usual, a tuple $(s,p,o) \in (\symAllURIs \cup \symAllBNodes) \times \symAllURIs \times (\symAllURIs \cup \symAllBNodes \cup \symAllLiterals)$ is an \emph{RDF triple}.
% A set of RDF triples is an \emph{RDF graph}.

{\RDFplus} extends such triples by permitting the embedding of another triple in the subject or object position of a given triple. We shall interpret the embedding triple as a statement about the embedded triple. An embedded triple may itself be a statement about a triple and, thus, may also contain embedded triples; and so forth.
	%Therefore, the model should allow for nesting.
%
	%As a basis for formalizing
	To formalize
such kind of (nested) embedding we
	%introduce
	define
a set $\symAllTRefsInfinite := (\symAllTRefsInfinite \cup \symAllURIs \cup \symAllBNodes) \times \symAllURIs \times (\symAllTRefsInfinite \cup \symAllURIs \cup \symAllBNodes \cup \symAllLiterals)$.
%
	As a basis for ruling
	%To rule
out infinitely deep nesting we introduce the notion of \emph{$k$-nestedness} (where $k$
	is an arbitrary
	%may be any
natural number): A tuple $r=(s,p,o) \in \symAllTRefsInfinite$ is $k$-nested if either 1.)~$k=0$ and $r\in (\symAllURIs \cup \symAllBNodes) \times \symAllURIs \times (\symAllURIs \cup \symAllBNodes \cup \symAllLiterals)$ or 2.)~$k>0$ and for all $x \in \bigl( \lbrace s,o \rbrace \cap \symAllTRefsInfinite \bigr)$ holds $x$ is $(k\!-\!1$)-nested.
%
Using $k$-nestedness we now
	%define
	introduce {\TRefLongVersion}s as
the new type of elements that may be used for triples in {\RDFplus}: A 
%
%Using $k$-nestedness we now define a
	%\emph{{\TRef}}
	\emph{{\TRefLongVersion}}, or \emph{\TRef} for short,
is a tuple $r = (s,p,o) \in \symAllTRefsInfinite$ for which exists a natural number $k$ such that $r$ is $k$-nested; to denote the set of all {\TRef}s we write $\symAllTRefs$, i.e.~$\symAllTRefs \subset \symAllTRefsInfinite$.

An \emph{{\RDFplusTriple}} is a tuple $(s,p,o) \in (\symAllTRefs \cup \symAllURIs \cup \symAllBNodes) \times \symAllURIs \times (\symAllTRefs \cup \symAllURIs \cup \symAllBNodes \cup \symAllLiterals)$.
A set
	%of {\RDFplusTriple}s
	of~{\RDFplusTriple}s  %% Layout Adjustment
is an \emph{{\RDFplusGraph}}.
We call an {\RDFplusTriple}
	%that contains a triple reference (in its subject or object position)
	$(s,p,o)$ for which $s \in \symAllTRefs$ or $o \in \symAllTRefs$,
a \emph{metadata triple}; note that any other {\RDFplusTriple} is an ordinary RDF triple.
For any {\RDFplusTriple} $t=(s,p,o)$ we define
	%$\fctTerms{t} := \lbrace s,p,o \rbrace$ and $\fctTRef{t} := \fctTerms{t} \cap \symAllTRefs$.
	$\fctTerms{t} := \lbrace s,p,o \rbrace$.
Overloading
	%functions $\symTermsFct$ and $\symTRefFct$, we write $\fctTerms{G} := \bigcup_{t \in G} \fctTerms{t}$ and $\fctTRef{G} := \fctTerms{G} \cap \symAllTRefs$ for any set $G$ of {\RDFplusTriple}s.
	function $\symTermsFct$, for any {\RDFplusGraph} $G$ we define $\fctTerms{G} := \bigcup_{t \in G} \fctTerms{t}$; furthermore, $\fctTRef{G} := \fctTerms{G} \cap \symAllTRefs$.

While syntactically {\RDFplusTriple}s and {\TRef}s are the same, they represent different concepts: {\RDFplusTriple}s may be interpreted as statements (similar to RDF triples) whereas a {\TRef} denotes an {\RDFplusTriple}
	%that is embedded in other {\RDFplusTriple}s.
	for the purpose of making a statement about it.
%
	\removable{Hence, trefs may be understood as a particular type of name for {\RDFplusTriple}s.}
However, since {\TRef}s carry all elements of the {\RDFplusTriple}s they denote, it is possible to obtain such a triple from its reference: We introduce a bijective function $\symDerefFct$ that maps the set of {\TRef}s to the set of all {\RDFplusTriple}s and that is defined as follows: Let $r=(s,p,o) \in \symAllTRefs$ be an arbitrary {\TRef}, then $\fctDeref{r}$ is the {\RDFplusTriple} $(s,p,o)$.

\subsection{Interpretation}

\todo{introduce this subsection (explain the purpose of the conversion function defined here)}

For any RDF triple $(s,p,o) \in (\symAllURIs \cup \symAllBNodes) \times \symAllURIs \times (\symAllURIs \cup \symAllBNodes \cup \symAllLiterals)$ and blank node $b \in \symAllBNodes$ we define:
\begin{align*}
	\mathrm{reif}\bigl( s,p,o, b \bigr) := \big\lbrace
		& (b,\mathTinyRDF{rdf\!:\!type},\mathTinyRDF{rdf\!:\!Statement}),
		(b,\mathTinyRDF{rdf\!:\!subject},s), \\
		& (b,\mathTinyRDF{rdf\!:\!predicate},p),
		(b,\mathTinyRDF{rdf\!:\!object},o)
	\big\rbrace
\end{align*}

\noindent
Let $G$ be a set of {\RDFplusTriple}s. For converting $G$ into a set of RDF triples we assume a bijective function $id : \fctTRef{G} \rightarrow B$ where $B \subset \symAllBNodes$ is a set of blank nodes such that $| B | = | \fctTRef{G} |$ and $B \cap \fctTerms{G} = \emptyset$. Hence, $id$ associates each {\TRef} in $G$ with a fresh blank node (not used in $G$). Furthermore, we introduce
	%an injective function $id^* : \fctTerms{G} \rightarrow \fctTerms{G} \cup B$
	another bijective function $id^*\! : \fctTerms{G} \rightarrow B \cup \bigl( \fctTerms{G} \!\setminus\! \fctTRef{G} \bigr)$
that is defined as follows: Let $x \in \fctTerms{G}$. If $x \in \symAllTRefs$ then $id^*\!(x) := id(x)$; if $x \notin \symAllTRefs$ then $id^*\!(x) := x$.
%
	Using $id^*$,
	%Overloading function $\mathrm{reif}$,
for any {\TRef} $r = (s,p,o) \in \fctTRef{G}$ we define:
% \vspace{-3mm} % Layout Adjustment
\begin{equation*}
	\mathrm{reif}( r ) := \mathrm{reif}\bigl( \, id^*\!(s), id^*\!(p), id^*\!(o), id^*\!(r) \, \bigr)
\end{equation*}

\noindent
For any {\RDFplusTriple} $t = (s,p,o) \in G$ we define:
\begin{equation*}
	\fctDescent{t} := \begin{cases}
\vspace{1mm}
		\big\lbrace (id(s),p,o) \big\rbrace \cup \mathrm{reif}(s) \cup \symDescentFct\bigl(\fctDeref{s}\bigr) & \text{if $s \in \symAllTRefs$ and $o \notin \symAllTRefs$,} \\
\vspace{1mm}
		\big\lbrace (s,p,id(o)) \big\rbrace \cup \mathrm{reif}(o) \cup \symDescentFct\bigl(\fctDeref{o}\bigr) & \text{if $s \notin \symAllTRefs$ and $o \in \symAllTRefs$,} \\
\vspace{1mm}
		\begin{minipage}[c]{65mm}\vspace{1mm}
			$\big\lbrace (id(s),p,id(o)) \big\rbrace \cup \mathrm{reif}(s) \cup \mathrm{reif}(o)$ \\
			\hspace*{15mm}$\cup \, \symDescentFct\bigl(\fctDeref{s}\bigr) \cup \symDescentFct\bigl(\fctDeref{o}\bigr)$
		\end{minipage} & \text{if $s \in \symAllTRefs$ and $o \notin \symAllTRefs$,} \\
		\big\lbrace (s,p,o) \big\rbrace & \text{else} .
\end{cases}
\end{equation*}

\noindent
Finally:
\begin{equation*}
	\fctDescent{G} := \bigcup_{t \in G} \fctDescent{t}
\end{equation*}

\begin{example}
	Let $G_\mathsf{ex} = \lbrace t_\mathsf{ex} \rbrace$ be an {\RDFplusGraph} that consists of a single {\RDFplusTriple} $t_\mathsf{ex} = \bigl( (\mathTinyRDF{ex:olaf},\mathTinyRDF{foaf:name},\mathTinyRDF{"Olaf"}), \mathTinyRDF{dct:source}, \mathTinyRDF{ex:olaf} \bigr)$.
% 	The result of converting $G_\mathsf{ex}$ to an RDF graph is:
% 	\begin{align*}
% 		\fctDescent{G_\mathsf{ex}} = \big\lbrace
% 			& (\mathTinyRDF{ex:olaf},\mathTinyRDF{foaf:name},\mathTinyRDF{"Olaf"}),
% 			(b, \mathTinyRDF{dct:source}, \mathTinyRDF{ex:olaf}), \\
% 			& (b,\mathTinyRDF{rdf\!:\!type},\mathTinyRDF{rdf\!:\!Statement}),
% 			(b,\mathTinyRDF{rdf\!:\!subject},\mathTinyRDF{ex:olaf}), \\
% 			& (b,\mathTinyRDF{rdf\!:\!predicate},\mathTinyRDF{foaf:name}),
% 			(b,\mathTinyRDF{rdf\!:\!object},\mathTinyRDF{"Olaf"})
% 		\big\rbrace
% 	\end{align*}
	The result of converting $G_\mathsf{ex}$ to an RDF graph $\fctDescent{G_\mathsf{ex}}$ is (in Turtle notation, prefix declarations omitted):
	\begin{footnotesize}%
	\begin{verbatim}
		 ex:olaf foaf:name "Olaf" .
		 _:b dct:source ex:olaf ; rdf:type rdf:Statement ;
		     rdf:subject ex:olaf ; rdf:predicate foaf:name ; rdf:object "Olaf" .
	\end{verbatim}%
	\end{footnotesize}
\end{example}

\noindent
Based on the conversion function, it is possible to interpret any {\RDFplus} model via the usual RDF semantics as defined in~\cite{Hayes04:RDFsemantics}.


\subsection{Representation}
\todo{briefly introduce our extension for Turtle and explain how this extension is related to {\RDFplus} and how it should be parsed by systems that do not ``understand'' {\RDFplus}}

	\begin{footnotesize}%
	\begin{verbatim}
		 <<ex:olaf foaf:name "Olaf">> dct:source ex:olaf .
	\end{verbatim}%
	\end{footnotesize}

\todo{From a didactic point of view it may be better to move this subsection to the beginning of Section~\ref{Section:DataModel}}
\section{Querying} \label{Section:Querying}

To fully benefit from our extended data model we extend the RDF query language SPARQL. Our extension, which we call \emph{\SPARQLplus}, adds new features that enable users to directly access metadata triples in queries. In particular, {\SPARQLplus} introduces a possibility for binding {\RDFplusTriple}s to query variables; such a variable may then be used in a triple pattern in order to ask for matching metadata triples. Furthermore, as a shortcut, (recursively nested) triple patterns may be directly embedded in triple patterns.

In this section we define {\SPARQLplus} formally. Due to space limitations, we focus on a core fragment of the language; this fragment corresponds to the core fragment of SPARQL that P\'{e}rez et al.~discuss in~\cite{Perez09:SemanticsAndComplexityOfSPARQL}. We also adapt P\'{e}rez et al.'s formalization approach, that is, we define {\SPARQLplus} using an algebraic syntax. In what follows we first introduce the syntax and then specify a compositional set semantics for {\SPARQLplus} queries. While an implementation of this semantics requires a system that supports our {\RDFplus} data model, we also aim to enable existing SPARQL execution systems to answer
	\removable{metadata-related}
{\SPARQLplus} queries over ordinary RDF data (in which statement-level metadata is represented via RDF reification). Consequently, we complement the definition of {\SPARQLplus}
	%in this section
with a function for converting {\SPARQLplus} queries to SPARQL queries. % and we show that the resulting SPARQL queries produce the same answer 


\subsection{Syntax of {\SPARQLplus}}

	Let $\symAllTPsInfinite := (\symAllVariables \cup \symAllTPsInfinite \cup \symAllURIs) \times (\symAllVariables \cup \symAllURIs) \times (\symAllVariables \cup \symAllTPsInfinite \cup \symAllURIs \cup \symAllLiterals)$ where $\symAllVariables$ is a set of query variables that is disjoint from $\symAllURIs$, $\symAllBNodes$, and $\symAllLiterals$, respectively.
	%We assume a set $\symAllVariables$ of query variables that is disjoint from $\symAllURIs$, $\symAllBNodes$, and $\symAllLiterals$, respectively. Let $\symAllTPsInfinite := (\symAllVariables \cup \symAllTPsInfinite \cup \symAllURIs) \times (\symAllVariables \cup \symAllURIs) \times (\symAllVariables \cup \symAllTPsInfinite \cup \symAllURIs \cup \symAllLiterals)$.
A \emph{\TPplus} is a tuple $tp \in \symAllTPsInfinite$ for which exists a natural number $k$ such that $tp$ is $k$-nested\footnote{$k$-nestedness for $\symAllTPsInfinite$ is defined analog to $k$-nestedness for $\symAllTriplesInfinite$.}; to denote the set of all {\TPplus}s we write $\symAllTPs$ (i.e.~$\symAllTPs \subset \symAllTPsInfinite$).

	We now define \emph{{\SPARQLplus} expression}s recursively \removable{as follows}:
	%\emph{{\SPARQLplus} expression}s are defined recursively \removable{as follows}:
\begin{enumerate}
	\item A \emph{\TPplus} is a {\SPARQLplus} expression.
	\item Let $tp \in \symAllTPs$ and $?v \in \symAllVariables$, then $(tp \OpAS ?v)$ is a {\SPARQLplus} expression.
	\item Let $P_1$ and $P_2$ be {\SPARQLplus} expressions and $R$ a filter condition%
		\footnote{For a formal definition of
			\removable{the syntax and semantics of}
		filter conditions we refer to~\cite{Perez09:SemanticsAndComplexityOfSPARQL}.}%
	, then $(P_1 \OpAND P_2)$, $(P_1 \OpUNION P_2)$, $(P_1 \OpOPT P_2)$, and $(P_1 \OpFILTER R)$ are {\SPARQLplus} expressions.
% 	\item Let $P$ be a {\SPARQLplus} expression and $S \subset \symAllVariables$ be a finite set of variables, then $\OpSELECT_S(P)$ is a {\SPARQLplus} expression.
\end{enumerate}

\begin{note}
		We note that any
		%Any
	(ordinary) SPARQL expression (as defined by P\'{e}rez et al.~\cite{Perez09:SemanticsAndComplexityOfSPARQL}) is a {\SPARQLplus} expression.
\end{note}

\noindent
For any {\SPARQLplus} expression $P$ we write $\fctTRefs{P}$ to denote the set of all {\TPplus}s embedded in a {\TPplus} in $P$; and $\fctVarsPlus{P}$ denotes the set of all variables mentioned in $P$ (including in embedded {\TPplus}s). Formally, if $P$ is a {\TPplus} $tp \!=\! (s,p,o)$,
then $\fctVarsPlus{tp} := \bigl( \lbrace s,p,o \rbrace \cap \symAllVariables \bigr) \cup \big\lbrace ?v \in \fctVarsPlus{x} \,\big|\, x \in \lbrace s,o \rbrace \cap \symAllTPs \big\rbrace$
and $\fctTRefs{tp} := \bigl( \lbrace s,o \rbrace \cap \symAllTPs \bigr) \cup \big\lbrace tp \in \fctTRefs{x} \,\big|\, x \in \lbrace s,o \rbrace \cap \symAllTPs \big\rbrace$.
If $P$ is not a {\TPplus}, then $\fctVarsPlus{P}$ and $\fctTRefs{P}$ are defined as follows:
	\par \vspace{1pt}
	\parbox{0.95\linewidth}{\centering
		\begin{tabular}{|c||c|c|} \hline
			{\SPARQLplus} expression $P$ &
			$\fctVarsPlus{P}$ &
			$\fctTRefs{P}$
			\\ \hline
			\hline
% 			{\TPplus} $(s,p,o)$ &
% 			$\bigl( \lbrace s,p,o \rbrace \cap \symAllVariables \bigr) \cup \big\lbrace ?v \in \fctVarsPlus{x} \,\big|\, x \in \lbrace s,o \rbrace \cap \symAllTPs \big\rbrace$ &
% 			$\bigl( \lbrace s,o \rbrace \cap \symAllTPs \bigr) \cup \big\lbrace tp \in \fctTRefs{x} \,\big|\, x \in \lbrace s,o \rbrace \cap \symAllTPs \big\rbrace$
% 			\\ \hline
			$(tp \OpAS ?v)$ &
			$\fctVarsPlus{tp} \cup \lbrace ?v \rbrace$ &
			$\fctTRefs{tp} \cup \lbrace tp \rbrace$
			\\ \hline
			$(P_1 \OpAND P_2)$ &
			$\fctVarsPlus{P_1} \cup \fctVarsPlus{P_2}$ &
			$\fctTRefs{P_1} \cup \fctTRefs{P_2}$
			\\ \hline
			$(P_1 \OpOPT P_2)$ &
			$\fctVarsPlus{P_1} \cup \fctVarsPlus{P_2}$ &
			$\fctTRefs{P_1} \cup \fctTRefs{P_2}$
			\\ \hline
			$(P_1 \OpUNION P_2)$ &
			$\fctVarsPlus{P_1} \cup \fctVarsPlus{P_2}$ &
			$\fctTRefs{P_1} \cup \fctTRefs{P_2}$
			\\ \hline
			$(P' \OpFILTER R)$ &
			$\fctVarsPlus{P'} \cup \fctVars{R}$ &
			$\fctTRefs{P'}$
			\\ \hline
% 			$\OpSELECT_S(P')$ &
% 			$\fctVarsPlus{P'} \cup S$ &
% 			$\fctTRefs{P'}$
% 			\\ \hline
		\end{tabular}
	}
	\par


\subsection{({\RDFplus} Based) Semantics of {\SPARQLplus}}
To define the semantics of {\SPARQLplus} we introduce \emph{valuation}s, that are, partial mappings $\eta : \symAllVariables \rightarrow \symAllTriples \cup \symAllURIs \cup \symAllBNodes \cup \symAllLiterals$.
Two valuations $\eta$ and $\eta'$ are \emph{compatible}, denoted by $\eta \compatible \eta'$, if for all $?v \in \fctDom{\eta} \cap \fctDom{\eta'}$ holds $\eta(?v) = \eta'(?v)$; note, $\fctDom{\eta}$ denotes the domain of a valuation $\eta$.

Let $\Omega_1$, $\Omega_2$ and $\Omega$ be sets of
	%valuations; let
	valuations and let
$R$ be a filter condition%
	%; and let $S \subset \symAllVariables$ be a finite set of variables.
	.
The algebra operators \emph{join}, \emph{union}, \emph{difference}, \emph{left outer-join},
	%\emph{selection}, and \emph{projection}
	and \emph{selection}
are defined as follows:
\begin{align*}
	\Omega_1 \Join \Omega_2 & := \lbrace \eta_1 \cup \eta_1 \,|\, \eta_1 \in \Omega_1 \text{ and } \eta_2 \in \Omega_2 \text{ and } \eta_1 \sim \eta_2 \rbrace
	\\
	\Omega_2 \cup \Omega_2 & := \lbrace \eta \,|\, \eta \in \Omega_1 \text{ or } \eta \in \Omega_2 \rbrace
	\\
	\Omega_1 \setminus \Omega_2 & := \lbrace \eta_1 \in \Omega_1 \,|\, \forall \eta_2 \in \Omega_2 : \eta_1 \not\sim \eta_2 \rbrace
	\\
	\Omega_1 \LJoin \Omega_2 &:= \left( \Omega_1 \Join \Omega_2 \right) \cup \left( \Omega_1 \setminus \Omega_2 \right)
	\\
	\sigma_R ( \Omega ) &:= \lbrace \eta \in \Omega \,|\, \eta \text{ satisfies } R \rbrace
% 	\\
% 	\pi_S ( \Omega ) &:= \lbrace \eta \,|\, \exists \, \eta' : \eta \cup \eta' \in \Omega \text{ and } \fctDom{\eta} \subseteq S \text{ and } \fctDom{\eta'} \cap S = \emptyset \rbrace
\end{align*}

\noindent
%For a valuation $\eta$ and a {\TPplus} $tp$, we write $\eta[tp]$ to denote the {\TPplus} that we obtain by replacing the variables in $tp$ (including potentially embedded {\TPplus}s) according to $\eta$. Note, if $\fctVarsPlus{tp} \subseteq \fctDom{\eta}$, then $\eta[tp]$ is an {\RDFplusTriple}.
%
	%Let $P$ be a {\SPARQLplus} expression and let $G$ be an {\RDFplusGraph}. The \emph{evaluation of $P$ over $G$},
	The \emph{evaluation} of {\SPARQLplus} expression $P$ over {\RDFplusGraph} $G$,
denoted by $\fctEvalPlus{P}{G}$, is defined recursively as follows:
\begin{enumerate}
	\item If $P$ is a {\TPplus} $tp$, then
		\begin{equation*}
			\fctEvalPlus{P}{G} := \big\lbrace \eta \,\big|\,
				\eta \text{ is a valuation with } \fctDom{\eta} = \fctVarsPlus{tp}
				\text{ and } \eta[tp] \in G \cup \fctTRefs{G}
			\big\rbrace
		\end{equation*}
		where $\eta[tp]$ denotes the {\TPplus} that we obtain by replacing the variables in $tp$ (including potentially embedded {\TPplus}s) according to $\eta$ (note, if $\fctVarsPlus{tp} \subseteq \fctDom{\eta}$, then $\eta[tp]$ is an {\RDFplusTriple}).
		\vspace{1ex} %% Layout Adjustment
	\item If $P$ is $(tp \OpAS ?v)$, then
		\begin{align*}
			\fctEvalPlus{P}{G} := \big\lbrace \eta \,\big|\,
				& \eta' \in \fctEvalPlus{tp}{G} 
				\text{ and } %\eta \text{ is a valuation with }
					\fctDom{\eta} = \fctDom{\eta'}\cup \lbrace ?v \rbrace
				\text{ and } \eta \compatible \eta' \text{ and }\\
				& \eta(?v) = \eta[tp] \,
			\big\rbrace
		\end{align*}
	\item If $P$ is $(P_1 \OpAND P_2)$, then $\fctEvalPlus{P}{G} := \fctEvalPlus{P_1}{G} \Join \fctEvalPlus{P_2}{G}$.
	\item If $P$ is $(P_1 \OpUNION P_2)$, then $\fctEvalPlus{P}{G} := \fctEvalPlus{P_1}{G} \cup \fctEvalPlus{P_2}{G}$.
	\item If $P$ is $(P_1 \OpOPT P_2)$, then $\fctEvalPlus{P}{G} := \fctEvalPlus{P_1}{G} \LJoin \fctEvalPlus{P_2}{G}$.
	\item If $P$ is $(P' \OpFILTER R)$, then $\fctEvalPlus{P}{G} := \sigma_R\bigl( \fctEvalPlus{P'}{G} \bigr)$.
% 	\item If $P$ is $\OpSELECT_S(P')$, then $\fctEvalPlus{P}{G} := \pi_S\bigl( \fctEvalPlus{P'}{G} \bigr)$.
\end{enumerate}

\noindent
% \removable{As usual, each valuation $\eta \in \fctEvalPlus{P}{G}$ is called a \emph{solution} for $P$ in $G$.}
% 
We emphasize that our definitions in this section resemble the definitions with which
	%P\'{e}rez et al.~define
the evaluation function $\fctEval{\cdot}{ }$ for ordinary SPARQL
	%expressions~\cite{Perez09:SemanticsAndComplexityOfSPARQL}.
	expressions is defined~\cite{Perez09:SemanticsAndComplexityOfSPARQL}.
As the primary difference we note that the definitions for SPARQL are based on so called \emph{solution mapping}s, that are, partial mappings $\mu : \symAllVariables \rightarrow \symAllURIs \cup \symAllBNodes \cup \symAllLiterals$, whereas we use valuations $\eta : \symAllVariables \rightarrow \symAllTriples \cup \symAllURIs \cup \symAllBNodes \cup \symAllLiterals$. However, any valuation
	%$\eta$ for which $\forall \, ?v \in \fctDom{\eta} : \eta(?v) \notin \symAllTriples$,
	that binds variables only to URIs, blank nodes, or literals (but not to {\RDFplusTriple}s),
may also be understood as a solution mapping. Based on this understanding, the following result shows that our query semantics for {\SPARQLplus} is fully compatible to the usual SPARQL semantics (defined in~\cite{Perez09:SemanticsAndComplexityOfSPARQL}), that is, any system that supports {\RDFplus} and {\SPARQLplus} may use {\SPARQLplus} semantics to answer (ordinary) SPARQL expressions over (ordinary) RDF graphs.

\begin{proposition} \label{Proposition:EquivalenceOfSemantics}
	For any
		%(ordinary)
	RDF graph $G$ and any
		%(ordinary)
	SPARQL
		%expression $P$
		expr.~$P$
	holds $\fctEvalPlus{P}{G} \!=\! \fctEval{P}{G}$.
\end{proposition}

% \todo{Motivate the following proposition (it is cool because it shows that we may reduce queried {\RDFplus} data to a minimum base and it nicely corresponds to Proposition~\ref{Proposition:JustificationForDataMinimality1}}
% 
% \begin{proposition} \label{Proposition:JustificationForDataMinimality2}
% 	For any {\RDFplusGraph} $G$ and any {\SPARQLplus} expression $P$ it holds $\fctEvalPlus{P}{G} = \fctEvalPlus{P}{G^+}$ where $G^+ = G \cup \fctTRefs{G}$.
% \end{proposition}
% 
% \todo{There is a minimal $G_\mathsf{min} \subseteq G$ with $G_\mathsf{min} = G^+$ ...}


\subsection{Using {\SPARQLplus} for Querying RDF Reification Based Metadata}

In addition to the {\SPARQLplus} semantics defined in the previous section, we now define
	%an alternative semantics for {\SPARQLplus}. This alternative semantics
	a function for converting {\SPARQLplus} queries to SPARQL queries. This conversion function
enables RDF systems that support the ordinary RDF data model (but not our {\RDFplus} data model), to answer {\SPARQLplus} queries.


Let $P$ be a {\SPARQLplus} expression.
	%We
	For converting $P$ to a SPARQL expression we
assume an \removable{(arbitrary)} injective function $v : \fctTRefs{P} \rightarrow \symAllVariables \cap \fctVarsPlus{P}$. Furthermore, we introduce another injective function $w : (\symAllVariables \cup \fctTRefs{P} \cup \symAllURIs \cup \symAllLiterals) \rightarrow (\symAllVariables \cup \symAllURIs \cup \symAllLiterals)$ that is defined as follows: Let $x \in (\symAllVariables \cup \fctTRefs{P} \cup \symAllURIs \cup \symAllLiterals)$. If $x \in \fctTRefs{P}$, then $w(x) := v(x)$; if $v \notin \fctTRefs{P}$, then $w(x) := x$.
%
Using $w$, for any $rp = (s,p,o) \in \fctTRefs{P}$ we write $\mathrm{reif}^w\!( rp )$ to denote
	the SPARQL expression $( ( ( tp_1 \OpAND tp_2 ) \OpAND tp_3 ) \OpAND tp_4 )$ where $tp_1$ is $\bigl(w(rp),\mathTinyRDF{rdf\!:\!type},\mathTinyRDF{rdf\!:\!Statement}\bigr)$, $tp_2$ is $\bigl(w(rp),\mathTinyRDF{rdf\!:\!subject},w(s)\bigr)$, $tp_3$ is $\bigl(w(rp),\mathTinyRDF{rdf\!:\!predicate},w(p)\bigr)$, and $tp_4$ is $\bigl(w(rp),\mathTinyRDF{rdf\!:\!object},w(o)\bigr)$.
	%following SPARQL expression: \begin{align*}
	%	\Bigl( \Bigl( \,\, \Bigl( & \bigl(w(rp),\mathTinyRDF{rdf\!:\!type},\mathTinyRDF{rdf\!:\!Statement}\bigr) \OpAND \bigl(w(rp),\mathTinyRDF{rdf\!:\!subject},w(s)\bigr) \Bigr) \OpAND \\
	%	& \bigl(w(rp),\mathTinyRDF{rdf\!:\!predicate},w(p)\bigr) \Bigr) \OpAND \bigl(w(rp),\mathTinyRDF{rdf\!:\!object},w(o)\bigr) \Bigr)
	%\end{align*}
%
Now, the \emph{$w$-based, RDF compatible expression} of $P$, denoted by $\fctDescentWP{w}{P}$, is defined recursively: % as follows:
\begin{enumerate}
	\item If $P$ is a {\TPplus} $(s,p,o)$, and:
		\begin{enumerate}
			\item if $s \notin \symAllTPs$ and $o \notin \symAllTPs$, then $\fctDescentWP{w}{P}$ is $(s,p,o)$; or
			\item if $s \in \symAllTPs$ and $o \notin \symAllTPs$, then $\fctDescentWP{w}{P}$ is $\bigl( \bigl( P_{s1} \OpAND P_{s2} \bigr) \OpAND (s'\!,p,o) \bigr)$; or
			\item if $s \notin \symAllTPs$ and $o \in \symAllTPs$, then $\fctDescentWP{w}{P}$ is $\bigl( \bigl( P_{o1} \OpAND P_{o2} \bigr) \OpAND (s,p,o') \bigr)$; or
			\item else $\fctDescentWP{w}{P}$ is $\bigl( \bigl( \bigl( \bigl( P_{s1} \OpAND P_{s2} \bigr) \OpAND P_{o1} \bigr) \OpAND P_{o2} \bigr) \OpAND (s'\!,p,o') \bigr)$;
		\end{enumerate}
		where $s'$ is $w(s)$, $o'$ is $w(o)$, $P_{s1}$ is $\mathrm{reif}^w\!(s)$, $P_{s2}$ is $\fctDescentWP{w}{s}$, $P_{o1}$ is $\mathrm{reif}^w\!(o)$, and $P_{o2}$ is $\fctDescentWP{w}{o}$.
		\vspace{1ex}

	\item If $P$ is
			%$(tp \OpAS ?v)$ with $tp = (s,p,o)$,
			$\bigl( (s,p,o) \OpAS ?v \bigr)$,
		then $\fctDescentWP{w}{P}$ is $( ( ( tp_1 \OpAND tp_2 ) \OpAND tp_3 ) \OpAND tp_4 )$ where $tp_1$ is $\bigl(?v,\mathTinyRDF{rdf\!:\!type},\mathTinyRDF{rdf\!:\!Statement}\bigr)$, $tp_2$ is $\bigl(?v,\mathTinyRDF{rdf\!:\!subject},w(s)\bigr)$, $tp_3$ is $\bigl(?v,\mathTinyRDF{rdf\!:\!predicate},w(p)\bigr)$, and $tp_4$ is $\bigl(?v,\mathTinyRDF{rdf\!:\!object},w(o)\bigr)$.

% 	\item If $P$ is $(P_1 \OpAND P_2)$, $(P_1 \OpUNION P_2)$, $(P_1 \OpOPT P_2)$, $(P_1 \OpFILTER R)$, or $\OpSELECT_S(P_1)$, then $\fctDescentWP{w}{P}$ is $(P_1' \OpAND P_2')$, $(P_1' \OpUNION P_2')$, $(P_1' \OpOPT P_2')$, $(P_1' \OpFILTER R)$, and $\OpSELECT_S(P_1')$, respectively; where $P_1'$ is $\fctDescentWP{w}{P_1}$ and $P_2'$ is $\fctDescentWP{w}{P_2}$.

		\vspace{1ex}
	\item If $P$ is $(P_1 \OpAND P_2)$, $(P_1 \OpUNION P_2)$, $(P_1 \OpOPT P_2)$, or $(P_1 \OpFILTER R)$, then $\fctDescentWP{w}{P}$ is $(P_1' \OpAND P_2')$, $(P_1' \OpUNION P_2')$, $(P_1' \OpOPT P_2')$, and $(P_1' \OpFILTER R)$, respectively; where $P_1'$ is $\fctDescentWP{w}{P_1}$ and $P_2'$ is $\fctDescentWP{w}{P_2}$.

% 	\begin{tabular}{|c||c||c||c||c||c|} \hline
% 		$\boldsymbol{P}$ &
% 		$(P_1 \OpAND P_2)$ &
% 		$(P_1 \OpOPT P_2)$ &
% 		$(P_1 \OpUNION P_2)$ &
% 		$(P_1 \OpFILTER R)$ &
% 		$\OpSELECT_S(P_1)$
% 		\\ \hline
% 		$\boldsymbol{ \fctDescentWP{w}{P}}$ &
% 		$(P_1' \OpAND P_2')$ &
% 		$(P_1' \OpOPT P_2')$ &
% 		$(P_1' \OpUNION P_2')$ &
% 		$(P_1' \OpFILTER R)$ &
% 		$\OpSELECT_S(P_1')$
% 		\\ \hline
% 	\end{tabular}
%
% 	\begin{center}
% 	\begin{tabular}{|c|c|} \hline
% 		$\boldsymbol{P}$ & $\boldsymbol{ \fctDescentWP{w}{P}}$ \\ \hline \hline
% 		$(P_1 \OpAND P_2)$ & $(P_1' \OpAND P_2')$ \\ \hline
% 		$(P_1 \OpOPT P_2)$ & $(P_1' \OpOPT P_2')$ \\ \hline
% 	\end{tabular}
% 	\hspace{10mm}
% 	\begin{tabular}{|c|c|} \hline
% 		$\boldsymbol{P}$ & $\boldsymbol{ \fctDescentWP{w}{P}}$ \\ \hline \hline
% 		$(P_1 \OpUNION P_2)$ & $(P_1' \OpUNION P_2')$ \\ \hline
% 		$(P_1 \OpFILTER R)$ & $(P_1' \OpFILTER R)$ \\ \hline
% 	\end{tabular}
% 	\end{center}
\end{enumerate}

\noindent
It is easy to see that our conversion function $\fctDescentWP{w}{P}$ replaces any \SPARQLplus-spe\-cif\-ic syntax element in $P$ by elements from the usual SPARQL syntax (defined in~\cite{Perez09:SemanticsAndComplexityOfSPARQL}). Thus:

\begin{proposition} \label{Proposition:EquivalenceOfSyntaxAfterConverting}
	For any {\SPARQLplus} expression $P$ any ($w$-based) RDF compatible expression of $P$ is an ordinary SPARQL expression (for all possible $w$).
\end{proposition}

\noindent
To show a (semantic) equivalence between {\SPARQLplus} expressions and the corresponding SPARQL expressions we need an additional algebra operator to project out (bindings for) query variables that the conversion introduces. We define \emph{projection} as follows: Let $\Omega$ be a set of valuations and let $S \subset \symAllVariables$ be a finite set of variables, then $\pi_S ( \Omega ) := \lbrace \eta \,|\, \exists \, \eta' : \eta \cup \eta' \in \Omega \text{ and } \fctDom{\eta} \subseteq S \text{ and } \fctDom{\eta'} \cap S = \emptyset \rbrace$.
Using projection we may now show the following equivalence:

\begin{proposition} \label{Proposition:QueryEquivalenceAfterConverting}
	For any {\RDFplusGraph} $G$ and any {\SPARQLplus} expression $P$ that does not contain a sub-expression
		\removable{of the form}
	$(tp \OpAS ?v)$ it holds $\fctEvalPlus{P}{G} = \pi_{\fctVarsPlus{P}}\bigl( \fctEvalPlus{P'}{\fctDescent{G}} \bigr)$ where $P'$ is an arbitrary RDF compatible expression $\fctDescentWP{w}{P}$ of $P$.
\end{proposition}

\noindent
The following result is a direct consequence of Propositions~\ref{Proposition:EquivalenceOfSemantics}, \ref{Proposition:EquivalenceOfSyntaxAfterConverting}, and~\ref{Proposition:QueryEquivalenceAfterConverting}:
\begin{corollary}
	For any {\RDFplusGraph} $G$ and any {\SPARQLplus} expression $P$ that does not contain a sub-expression
		\removable{of the form}
	$(tp \OpAS ?v)$ it holds $\fctEvalPlus{P}{G} = \pi_{\fctVarsPlus{P}}\bigl( \fctEval{P'}{\fctDescent{G}} \bigr)$ where $P'$ is an arbitrary RDF compatible expression $\fctDescentWP{w}{P}$ of $P$.
\end{corollary}

\noindent
\todo{briefly discuss the Corollary}
\section{Experiments}
Show the benefits in experiments
Systems
-	Bigdata
-	Virtuoso

Data
-	Datasets + artificially generated meta-level statements
-	Size of the data
-	Size of the indexes 
Loading Time Performance
Querying Time Performance


\section{Conclusion} \label{Section:Conclusion}
-	Multiple vendors implemented the proposal:
o	SYSTAP, LLC (Bigdata)
o	OpenLink (Virtuoso)
-	Several organizations have agreed to to support the proposal.
-	Future work
o	adoption in next version of RDF and SPARQL
o	Standardize
o	Prepare W3C Member Submission:



%\begin{appendix}
%\input{appendix}
%\end{appendix}
%
\bibliographystyle{abbrv}
\bibliography{paper}
\end{document}
