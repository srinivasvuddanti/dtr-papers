\section{Related Work}
Overview of related work and their problems / reasons why not adopted in commercial solutions, e.g. 
Reification
Reification builds a model of a statement:
_:s1 rdf:subject   :SAP .
_:s1 rdf:predicate :bought .
_:s1 rdf:object    :sybase .
_:s1 rdf:type      rdf:Statement .
_:s1 dc:source     reuters:us-sybase .

-	Has a very bad reputation
-	Extremely inefficient 
-	5x the data on disk when you reify statements.
-	Awkward syntax
-	Cognitively challenging to write queries against reified data
-	Rarely used.
Other approaches to be discussed:
-	Bernhard Schueler, Sergej Sizov, Steffen Staab, Duc Thanh Tran: Querying for meta knowledge. WWW 2008:625-634
-	Thanh Tran, Peter Haase, Boris Motik, Bernardo Cuenca Grau, Ian Horrocks: Metalevel Information in Ontology-Based Applications. AAAI 2008:1237-1242
-	Nuno Lopes, Axel Polleres, Umberto Straccia, Antoine Zimmermann: AnQL: SPARQLing Up Annotated RDFS. International Semantic Web Conference 2010:518-533
-	Antoine Zimmermann, Nuno Lopes, Axel Polleres, Umberto Straccia: A general framework for representing, reasoning and querying with annotated Semantic Web data. J. Web Sem. 11: 72-95 (2012)
-	Umberto Straccia, Nuno Lopes, Gergely Lukacsy, Axel Polleres: A General Framework for Representing and Reasoning with Annotated Semantic Web Data. AAAI 2010
-	Jeremy J. Carroll, Christian Bizer, Patrick J. Hayes, Patrick Stickler: Named graphs. J. Web Sem. (WS) 3(4):247-267 (2005)
-	Paula Severi, Jos� Luiz Fiadeiro, David Ekserdjian: Guiding Reification in OWL through Aggregation. Description Logics 2010
-	Nicole Alexander, Siva Ravada: RDF Object Type and Reification in the Database. ICDE 2006:93
Solution & Contributions
-	Efficient management of meta-level statements 
o	Avoids the 5x growth in data associated with RDF Reification
o	Fast query answering
-	Convenient syntax for data and queries
o	Proposed extensions for SPARQL, Turtle
-	Harmonized with RDF Reification
o	Existing data and queries still work
o	Extension to SPARQL Semantics and Algebra
-	Can be combined with named graphs
o	Both methods can be mixed in the same database.

