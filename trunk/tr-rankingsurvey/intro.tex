\section{Introduction}\label{sec:intro}
Scenarios characterized by searching and browsing on large volumes of data or documents require of special treatment in order to guide the users to the most relevant pieces of information. Typically, users have to select and filter all the information they go through until they find a relevant piece of data that matches what they are looking for. Also, user behavior studies have found out that users in Web search engines are viewing fewer result pages (Jansen \& Spink, 2006), which evidences the importance of ranking outcomes.

In the traditional Web the information space is modeled as a corpus of documents that establish links among them as an implicit way to state relationships within the information they contain. Users can make use of these links to navigate the information moving from one document to another using Web browsers. Following this model, referred to as the Web of documents, search engines were proposed as a way to facilitate the navigation towards finding the required information, and retrieval mechanisms have been devised that make use of known properties of the link structure (Broder et al., 2000), being a notable example the PageRank algorithm (Brin \& Page, 1998). Despite current Web document retrieval solutions have demonstrated to be useful, new challenges appear when dealing with finer-grained information spaces where entities formally described and the relationships among them play the main role, and not the documents where they appear or are mentioned in (Sheth, Budak Arpinar \& Kashyap, 2004). New methods for exploiting semantic relationships between data must be considered in order to make the most out of the information usage. These ideas are used and applied in the context of what is called the “Web of Data”, described in (Bizer, Heath, \& Berners-Lee, 2009) as “a Web of things in the world”, in contrast to the traditional abovementioned Web of documents. Basically, what favors the trend from the Web of documents to the Web of Data relies on the limitations of human capabilities for consuming huge amounts of information and the need for data. This, together with the improvements on machine’s power, helps to process the information and convert it into data ready for direct consumption. Furthermore, converting Web documents (unstructured data) to data (structured) helps to achieve data and service integration purposes. In what follows we describe the main elements of the Linked Data initiative\footnote{http://linkeddata.org/} as it can be considered the cornerstone of the Web of Data nowadays.

In the last decade, methodologies from database, artificial intelligence, information retrieval and linguistics research have been combined under the idea of pursuing a Semantic Web that helps to overcome the challenge of dealing with vast amounts of heterogeneous information (Lassila, 2007). All the efforts carried out to find a solution to this problem have produced different formalisms to model the knowledge implicitly contained in the information. Notably, the specification of the Resource Description Framework (Klyne \& Carroll, 2004), RDF Schema (McGuinness \& van Harmelen, 2004) and the Web Ontology Language (Brickley \& Guha, 2004) have been devised as languages for the representation of semantics. While having the required tools and capabilities to express the available knowledge, the fact of unifying all different perceptions of the real world under the same formal representation is still nowadays a challenge, due to the distributed nature of the Web that requires reconciling the semantics of disparate, heterogeneous schemas and representations. In order to overcome this problem approaches like the Linked Open Data initiative have arisen. As stated in (Bizer, Heath, \& Berners-Lee, 2009) ``Linked Data is simply about using the Web to create typed links between data from different sources''. In this way, the Web of Linked Data aims at building a dynamic set of data modeled using very simple principles while still keeping a common representation of the shared knowledge. As outlined in (Berners-Lee, 2006), the main principles of Linked Data are:
\begin{enumerate}
	\item Use URIs as names for things;
	\item Use HTTP URIs so that people can look up those names;
	\item When someone looks up an URI, provide useful information, using the standards (RDF, SPARQL);
	\item Include links to other URIs, so that they can discover more things.
\end{enumerate}

In addition, the main tasks that have to be performed in order to publish data as Linked Data are (i) to assign consistent URIs to data published, (ii) to generate links, and (iii) to publish metadata that allows further exploration and discovery of relevant datasets.

The Linked Data initiative has an enormous potential because it facilitates access to the very large amounts of information available on the Web in a structured and integrated fashion (Bizer, Heath, \& Berners-Lee, 2009). However, exploiting vast amounts of information requires new techniques that facilitate the user requirements for consuming and managing data. When searching for information, the fact of retrieving a significant collection of results satisfying the user requirements is very important, but the manner how these results are presented, filtered or ranked to the user can impact in a more important grade the way a user identifies the piece of information that better approximates to the target of his/her search. To help in this task ranking algorithms are used.

In a few words, a ranking algorithm implements a function that accepts a set of items and returns an ordered version of the set without modifying the items themselves. The function is implemented taking into account certain preferences that determine the order of the items. In this way, the same collection of items could be ranked following different approaches, i.e. different order functions. Whilst the area of information retrieval has addressed and provided different approaches for this problem, e.g. PageRank (Brin \& Page, 1998), HITS (Kleinberg, 1998) and SALSA (Lempel \& Moran, 2000), there is still a lack of consensus referring to the problem of ranking structured data as that exposed in the Web of Linked Data. As stated previously, when dealing with structured information, entities and the relationships among them play the main role, and not the documents where they appear.

The motivation of this work is to formalize the problem of ranking linked data and give a comprehensive overview of existent ranking methods for the Web of Data. There are other survey studies concerning to the topic of semantic Web search (Hildebrand, van Ossenbruggen \& Hardman 2007; Mäkelä, 2007; Want, 2008), where ranking algorithms for structured data are to some extent described. However, to the best of our knowledge none of the existing works gives a complete overview of ranking methodologies for the Web of Data that helps to understand the benefits and drawbacks of each one. This is of great importance for the future of the Web of Linked Data, as the same problems of volume that appear in the Web will arise as the Web of Data grows. The main target of this work focuses on helping researchers in the Semantic Web community to identify and understand the problem of ranking information. After a review of the literature, we have selected the most relevant algorithms according to their impact in this field. In this way, we have tried to homogenize the vocabulary employed with the aim of settling a common reference for semantic ranking methodologies.

\paragraph{Outline}
