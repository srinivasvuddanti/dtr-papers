\section{Popularity}
Wikipedia defines popularity\footnote{http://en.wikipedia.org/wiki/Popularity} as the quality of being well-liked or common, or having a 
high social status. Translating this to information systems, popularity ranking algorithms rely on link analysis to compute the relevance of a 
node of the graph in terms of its input and output linkage.
With the arrival of the Web, link analysis was proposed as a new method to rank the hypertext documents considering the way the information 
is represented and related \cite{}(Getoor and Diehl, 2005). Unlike content-based (statistics-based?) ranking, link analysis strategies try to incorporate 
structural features of the information during the ranking computation. Link analysis is a technique relying on examining the graph structure 
established among items, where the nodes of the graph are the items to rank and the edges are the relationships or links among items. By inspecting 
the graph structure implicit properties can be derived and included in the ranking process. Link analysis can be thought as a case of success, 
which was originally implemented in algorithms like PageRank \cite{}(Brin \& Page, 1998), HITS \cite{}(Kleinberg, 1998) and SALSA \cite{}(Lempel \& Moran, 2000), 
commercially exploited by the most popular search engines. Inspired by this philosophy several extensions have been developed that increase 
the corpus of link analysis methodologies to deal with structured and semi-structured information\footnote{This classification has been taken from 
\cite{}(Delbru, Toupikov, Catasta, Tummarello \& Decker, 2010)}, namely:
\begin{itemize}
\item Weighted link analysis. The aim of this technique is to assign more relevance to certain kind of links depending on its type during the ranking 
computation. A major challenge is how to assign the weight to the links without having negative performance implications. Most of the approaches 
under this classification were proposed in the topic of database research, and for this reason they are not directly applicable on web-scale. 
As an example for this category we can consider the works described in \cite{}(Xing \& Ghorbani, 2004) and \cite{}(Baeza-Yates \& Davis, 2004).
\item Hierarchical link analysis. This technique performs a layered exploration of the underlying data and it is intended for distributed environments. 
For example, first considering relationships among super nodes or datasets and secondly considering relationships among resources. An example for this 
category can be found in \cite{}(Xue, et al. 2005).
\item Semantic Web link analysis. This family of methods tries to exploit the semantic of relationships during the ranking process. This technique can 
be thought as an evolution of the weighted link analysis applied to the Semantic Web context. As an example we can consider the algorithms described 
in \cite{}(Ding et al., 2004) and \cite{}(Anyanwu, Maduko, \& Sheth, 2005), being the last one the culmination of the ideas previously introduced in 
\cite{}(Anyanwu \& Sheth, 2002) and \cite{}(Anyanwu \& Sheth, 2003). Examples of early commercial applications exploiting semantic Web link analysis include the 
works described in \cite{}(Sheth, Avant \& Bertram, 2001), \cite{}(Avant et. al, 2002) and \cite{}(Sheth, 2005).
\end{itemize}

