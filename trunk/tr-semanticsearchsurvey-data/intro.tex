\section{Introduction}
% unsolved information needs%
Advancements in search technologies enable users to deal with and to exploit the ever increasing amount of information that we can find in different settings -- from the personal desktop to enterprises' databases up to the large-scale Web. As opposed to the structured querying capabilities commonly provided by commercial databases that are only accessible by experts, \emph{search} is rather understood as an end-user oriented paradigm that is based on intuitive interfaces and access mechanisms. Solutions that are widely used especially on the Web are based on keyword search, and in smaller-scale scenarios, also support natural language (NL) inputs. While they are easy to use, the inputs provided by the users through these interfaces are ambiguous and limited representations of the underlying information need (query intent) due to the limitations of the keyword search paradigm. Further, in most search scenarios not only the query, but also the data representation is lacking precise structure and semantics. Accordingly, the core problems to be addressed in search are (1) to \emph{understand the query intent and the data}, (2) to \emph{match query against data}, and (3) to \emph{rank results}. 

Modern search engines are efficient in retrieving documents (web pages in the case of web search) from large collections based on matching the text of the user's query with the textual content of the document or the words used to label the links to the document (\emph{anchor text}). These textual clues are often combined with additional features that measure the popularity of results based on usage data (\emph{click-log data}) or using metrics computed on the web graph such as \emph{PageRank} and its derivates. In combination, these relatively simple, but scalable matching techniques are able to address a large fraction of common information needs. They work particularly well when the user is able to explicitly name the target of the query, this reference is unambiguous and refers to a popular item widely referred to with that name. An example is a product search, where the user enters the name of the product and that product is widely known by that name, e.g \emph{canon ixus 115}, a digital camera.


\subsection{Unsolved Queries}

While search engines have been effective in addressing common needs, most queries are still in the ``long tail'' of Web query logs. Baeza-Yates et al. report that even when taken a full year of query logs 88\% of the unique queries are singleton queries, i.e. appear only once in the year \cite{baeza07sigir}. A lot of this variety comes from syntactic variations of the same popular need and can be addressed with spell-corrections and query suggestion techniques. However, the remaining queries tend to be less common because they refer to less popular needs that can not be easily addressed by popularity ranking (due to the sparsity of signals), and more likely to need disambiguation or to require complex processing.

%\begin{itemize}
%	\item 
	\textbf{Ambiguous Queries:} 
	Ambiguous queries arise naturally when the user may not be able to name precisely the item that is sought. Coming back to the example of product search, a user who is in the early stages of selection may not know exactly which product to look for. In this case, the user may name one or two characteristics of the product, such as \emph{cheap digital camera}. The concept of digital camera covers a broad range of actual entities, and there might be some discussion as to which of these can be considered cheap.There are also systemic factors that lead to an increase in ambiguity. Despite the expectation that longer queries should leave to better results, current search engines perform poorer on longer specifications of the same information need, largely because search engines employ conjunctive semantics (an implicit AND operator between words), and the space of documents that contain all the words becomes smaller with every word added and determining the important words becomes more difficult. Thus users have learned to avoid long queries, leading to shorter but more ambiguous queries. Lastly, users will try to find what they are looking for with the least possible words, adding additional terms after realizing that their original query was ambiguous. For example, when searching for people the user may type in a name, only to realize that there are multiple persons with the same name. Often, this is aggravated by the bias toward a single popular answer. For example, when searching for \emph{george bush}, the top results are crowded by references to the same person, even though this is a common name.\footnote{For other notable persons named George Bush, see \url{http://en.wikipedia.org/wiki/George_Bush_\%28disambiguation\%29}}
			
%	\item 
	\textbf{Complex Queries:} We include in this class all queries that go beyond a reference to a single named entity, and involve several entities and relationships between them. These queries arise again when the user does not know the answer to a question and therefore has to describe the item sought and its relationships to other items that he or she can name. For example, searching for this paper by the name of the second author is relatively straightforward. However, assuming the user does not know (or do not remember) the name, he or she might look for a \emph{semantic search survey paper written by a young researcher at yahoo}. Complex queries are particularly hard to address in the traditional information retrieval paradigm when they require information extraction and aggregation, i.e. when not all the entities and relationships are mentioned in a single document. Queries for events naturally involve multiple entities, typically the participant in the event and a reference to another item, e.g. \emph{barack obama birth certificate}. 
%\end{itemize}

Clearly, these two categories represent orthogonal aspects, namely the ambiguity of query terms and the complexity of the intended information need. Many queries in the long tail are both ambiguous and complex. They are characterized by the requirement for a \emph{higher level understanding} of the meaning of the query and data than the level of keywords. As we discuss later, some semantic search approaches use this higher level, conceptual understanding of queries and data to compose answers (via reasoning or computation) that do not even explicitly appear in the data. 

\subsection{Semantic Resources}

While the above problems are not new, the renewed interest in trying to address them can be explained by the increasing availability of semantic resources in terms of large data sets and schemas. Many semantic search systems have emerged specifically to exploit these resources. Here we review the main trends that have led to this development, and we return to describe the types of semantic data in more detail in Section~\ref{sec:data}. 

Semantic annotations of documents are available in increasing amounts due to advances in both Information Extraction and the adoption of standard formats for the semantic annotation of documents. The two developments are complementary in providing annotations for the Web. IE technology is particularly adept at extraction from large sites on the Web, where the investments in creating training data for supervised methods of learning extractors (e.g. wrapper induction) pay off quickly. On the other hand, publishers are increasingly motivated by search engine providers to provide explicit semantic annotations so that consumers do not need to apply extraction. Semantic annotations in this case are embedded inside HTML pages using standards such as RDFa (a W3C standard\footnote{http://www.w3.org/TR/xhtml-rdfa-primer/} for embedding RDF data in Web pages), microformats or microdata. The practice of annotating webpages using these formats have thus brought us closer to acquiring semantic annotations in the long tail of the Web. As a result of active adoption and support from Web search engines providers such as Google and Yahoo!, 10 percent of all Web pages are estimated to contain some form of semantic data markups such as RDFa~\cite{DBLP:conf/semweb/MikaMZ09}. 

Complementary to these efforts, the Linking Open Data\footnote{\url{http://www.w3.org/wiki/SweoIG/TaskForces/CommunityProjects/LinkingOpenData}} movement that has emerged in recent years facilitated the process of publishing entire datasets (knowledge bases) on the Web. Large structured datasets that have been formerly kept in closed databases are now exposed as publicly Web-accessible RDF data. (For a list of datasets, see the Linked Data\footnote{\url{http://linkeddata.org/}} community site or the Data Hub \footnote{\url{http://thedatahub.org/}} maintained by CKAN.) This method of publishing is often used in the area of the sciences and the government where data is in the public domain. Efforts are under way to link these datasets by establishing links (references between entities) in different datasets. Besides these domain specific datasets, broad encyclopedic knowledge bases such as Dbpedia and Freebase are now also available for research. These resources are particularly valuable for supporting open domain search and query answering. Altogether, the Linked Data cloud is difficult to estimate in size but it contains billions of triples\footnote{\url{http://stats.lod2.eu/}}.

When making data available on the Web, publishers can choose the schema to publish their data. Many of the Linked Data datasets reflect the internal schema of the database that is being exposed. However, in this case the burden is on the consumer of the data to make sense of the schemas of the individual datasets and map them to a shared, global schema before being able to aggregate instance data. This type of schema matching is difficult and can not be fully automated in practice. A better approach is for publishers and consumers to agree on a shared schema and reuse that schema whenever possible. A recent example of this effort is the schema.org collaboration between the main search engine providers (currently, Bing, Google, Yahoo and Yandex). Schema.org provides a set of schemas for common types of objects on the Web, such as places, people, organizations, reviews, recipes etc.

The Semantic Web movement catalyzed the above developments by establishing a basic knowledge representation paradigm (RDF)\footnote{RDF stands for Resource Description Framework, a W3C Semantic Web standard for data representation and exchange on the Web: \url{http://www.w3.org/RDF/}}. RDF is flexible enough to capture any data that can be represented as a graph and has a number of serialization formats. RDF is commonly used to publish and manipulate semantic resources and is supported by a growing range of tools and APIs in various programming languages. In particular, the so called 'triple stores' that provide storage for RDF data have matured over the years and scale to several hundreds of millions triples (facts) on single machines.\footnote{For a benchmark of triple stores, see \url{http://www4.wiwiss.fu-berlin.de/bizer/BerlinSPARQLBenchmark/}} Triple stores, however, do not provide the kind of retrieval functionality described in this paper. Another Semantic Web standard, the Web Ontology Language (OWL) has standardized the way in which schemas are shared on the Web.

%Targeting these problematic cases, there is a category of search solutions -- often referred to as semantic search solutions -- that aim at exploiting the large and increasing amount of semantic resources that have been made available in different settings. Prominently, Google refers to this as the next generation of search in a recent issue of the Wall Street Journal.

%Semantic resources are used by these solutions for the interpretation of queries and data. They include taxonomies, thesauri and formal ontologies. More generally, also structured data exposed as graph-structured RDF\footnote{RDF stands for Resource Description Framework, a W3C Semantic Web standard for data representation and exchange on the Web: \url{http://www.w3.org/RDF/}} data are regarded as semantic resources (often called \emph{semantic data}). They come as standalone RDF datasets or are embedded in Web pages in the form of RDFa These data have been increasing rapidly. As a result of community projects such as Linking Open Data\footnote{\url{http://www.w3.org/wiki/SweoIG/TaskForces/CommunityProjects/LinkingOpenData}}, a large amount of structured datasets formerly kept protected in databases are now exposed as publicly Web-accessible RDF data (e.g. Linked Data\footnote{\url{http://linkeddata.org/}}). Besides large companies such as BestBuy and Facebook, also national governments such as the US and UK have followed this direction of publishing and linking semantic data on the Web. 

The main idea behind semantic search solutions is to use these semantic resources to improve the search performance. Taxonomies, thesauri, and ontologies capture semantics that can be used to understand the query and data and to \emph{resolve ambiguities}. Instead of returning textual data (Web pages, documents), precise facts capturing entity related information and their relationships can be directly retrieved from semantic data to provide \emph{direct answers to complex queries}. 

\subsection{Commercial Interest}

Capitalizing on the opportunities that arise from the already large and increasing amount of semantic resources on the Web, commercial search engines have shown increasing interest in Semantic Search. To call out some example, one of the first commercial search engines to employ semantic analysis was Powerset\footnote{\url{http://en.wikipedia.org/wiki/Powerset_(company)}}, which makes use of semantic data extracted from Wikipedia to answer complex questions. Yahoo! has pioneered the use of semantic annotations in Yahoo! SearchMonkey, which has been experimentally proven to help users judge the relevance of search results \cite{DBLP:conf/sigir/HaasMTB11}. The same method is also used by Google to provide rich result snippets\footnote{\url{http://www.google.com/webmasters/tools/richsnippets}}. Google also acquired Metaweb, the makers of Freebase, and widely published its intent to employ semantic search technology more extensively in the future\footnote{\url{http://on.wsj.com/w7OPag}}. As a last example, Facebook intends to extend its internal graph of people and other types of resources (e.g. organizations with Facebook pages) with additional knowledge from webpages. Facebook publishes a set of schemas under the name Open Graph Protocol \footnote{\url{http://ogp.me}} for developers who want to annotate their webpages. OGP annotations are also used by Facebook to interpret the type of object that the user is interested in when pressing a 'Like' button on a webpage.

\subsection{Research fragmentation}

Despite the commercial interest, the research effort in semantic search has been hampered by the lack of dissemination across research fields. As we will show in our survey, a wide range of semantic search approaches have been proposed by researchers from different communities, including Databases, Information Retrieval (IR) and Semantic Web. Correspondingly, semantic search has been viewed from different angles, resulting in a plethora of solutions. Existing approaches greatly vary in the \emph{data}, (2) the \emph{semantic resources}, the \emph{information needs} and (4) the \emph{querying paradigms} that are supported. Further, research work found in literatures often studies one of the specific (5) \emph{subproblems in search}: namely, different concepts and techniques have been proposed for (5a) the interpretation of query inputs and data, (5b) matching the query intent against data and (5c) ranking results.  


%Most notably, there are principle differences between semantic search approaches, which use semantic resources to improve document retrieval, and the ones that compute direct results, i.e. semantic data retrieval. That is, whereas the one category of approaches focuses on textual data, the other category targets the case of structured and semantic data. The specific types of semantic resources used by these approaches vary and may include thesauri, full-fledge ontologies or large amounts of semantic data. With respect to information needs, solutions focusing on the retrieval of single entities (and documents) can be distinguished from relational ones, where relationships between entities have to be considered. Substantial differences also exist between systems, which either use NL, keywords or facet-based operations as the primary means for querying. 

Our work unifies the research field of semantic search by determining the key areas of research, identifying the key contributions independent of the field in which they were published and placing them in a unified framework. Such an overview of a state-of-the-art will enable researchers and practitioners to design and develop semantic search systems based on best-of-breed methods. We also reflect on the evaluation and demonstrated performance of various approaches and discuss the challenges ahead.

There are few studies in related work that would compare different approaches to semantic search~\cite{DBLP:conf/semweb/KaufmannB07,DBLP:conf/esws/TranMH10}. However, this line of work focuses on the differences in querying paradigms. In a step towards a classification of semantic search approaches, Mangold discusses existing solutions along the dimensions of architecture, coupling, transparency, user context, ontology structure and ontology technology~\cite{DBLP:journals/ijmso/Mangold07}. The last two criteria basically capture the differences in semantic resources among existing systems, which are also discussed in our study. In addition to query paradigms and semantic resources, our study includes the aspects of data, information needs, and provides a fine-grained taxonomy that distinguishes approaches by the subproblems in search they addressing. Further, whereas the previous survey~\cite{DBLP:journals/ijmso/Mangold07} focuses on document retrieval, our study aims at a unified view of document and data retrieval. In particular, we cover a large body of recent work that is not part of that previous survey.  


%	\item Besides this general overview, we discuss the \emph{state-of-the-art in semantic data retrieval}, a direction of semantic search that aims at direct results as answers. For this, different data management and processing tasks are needed, from the crawling and integration of data up to the computation and presentation of results. 
%	\item We present specific solutions that have been proposed for these individual tasks. Particular emphasis is put on the most commonly used interface, namely \emph{keyword search}, and the concepts and techniques for dealing with the core search tasks of \emph{matching and ranking}. 
 


\subsection{Outline}
The paper is organized as follows. An overview of semantic search and the concepts it entails is presented in Section~\ref{sec:ss}. Then, the paper is organized mainly along the aspects based on which we distinguish different semantic search approaches. Different types of information needs, querying paradigms, data and semantic resources are presented in Section~\ref{sec:needs},~\ref{sec:querying},~\ref{sec:data} and ~\ref{sec:semantics}. Then approaches are discussed along the subproblems of data interpretation, query interpretation, matching and ranking in Section~\ref{sec:content},~\ref{sec:query},~\ref{sec:matching} and~\ref{sec:ranking}. After that, we present the main types of semantic search approaches and discuss their performances in Section~\ref{sec:approaches}, 
%Main challenges derived from that and implications for future work are presented in Section~\ref{sec:challenges}, 
before we finally conclude in Section~\ref{sec:conclusion}.